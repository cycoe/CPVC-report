%------------------氯化聚氯乙烯的热稳定和润滑体系研究-----------------%
%----------------------------朱浩南-----------------------------%

%---------------------------定义文档类型-------------------------%
%   \documentclass[param]{class}
%   param{
%       a4paper: A4纸
%       oneside: 单面打印
%       onecolumn: 单列排版
%       12pt: 默认字体大小
%   }
%   class{
%       report: 长报告、论文
%       article: 短文章
%       ctexrep: ctex report
%---------------------------------------------------------------%
    
\documentclass[a4paper, oneside, onecolumn, 12pt]{ctexrep}    %transmag 实现双列排版,摘要跨列

%----------------------------引入宏包---------------------------%
\usepackage[utf8]{inputenc}
\usepackage{xeCJK}      %CJK引擎
%\usepackage{abstract}  %摘要与关键词
\usepackage{graphicx}   %插图工具包
\usepackage{natbib}     %漂亮的bib引用
\usepackage{bm}         %设置缩进
\usepackage{fancyhdr}   %定制页眉页脚
\usepackage{chemfig}    %化学方程式
\usepackage{amsmath}    %数学公式阵列
\usepackage{multirow}   %表格合并单元格
\usepackage{setspace}   %行距宏包
\usepackage{geometry}   %页边距宏包
\usepackage{fontspec}   %字体
\usepackage{enumerate}  %定制列表序号样式
\usepackage{makecell}   %表格内容换行
\usepackage[justification=centering]{caption}   %标题设置

%----------------------------设置字体族--------------------------%
%\CJKfamily{SimSun}
\setmainfont{Times New Roman}
\setsansfont{Arial}
\setmonofont{Noto Mono}
\setCJKmainfont[BoldFont=SimHei]{STZhongsong}

%\newcommand{\song}{\CJKfamily{simsun}}  % 宋体
\newcommand{\hei}{\CJKfamily{SimHei}}   % 黑体


%------------------------------设置字体大小------------------------%  
\newcommand{\chuhao}{\fontsize{42pt}{\baselineskip}\selectfont}     %初号  
\newcommand{\xiaochuhao}{\fontsize{36pt}{\baselineskip}\selectfont} %小初号  
\newcommand{\yihao}{\fontsize{28pt}{\baselineskip}\selectfont}      %一号  
\newcommand{\erhao}{\fontsize{21pt}{\baselineskip}\selectfont}      %二号  
\newcommand{\xiaoerhao}{\fontsize{18pt}{\baselineskip}\selectfont}  %小二号  
\newcommand{\sanhao}{\fontsize{15.75pt}{\baselineskip}\selectfont}  %三号  
\newcommand{\sihao}{\fontsize{14pt}{\baselineskip}\selectfont}      %四号  
\newcommand{\xiaosihao}{\fontsize{12pt}{\baselineskip}\selectfont}  %小四号  
\newcommand{\wuhao}{\fontsize{10.5pt}{\baselineskip}\selectfont}    %五号  
\newcommand{\xiaowuhao}{\fontsize{9pt}{\baselineskip}\selectfont}   %小五号  
\newcommand{\liuhao}{\fontsize{7.875pt}{\baselineskip}\selectfont}  %六号  
\newcommand{\qihao}{\fontsize{5.25pt}{\baselineskip}\selectfont}    %七号

%------------------------------标题名称定制-------------------------------%
%\renewcommand{\contentsname}{目录}  % 将Contents改为目录
%\renewcommand{\abstractname}{摘要}  % 将Abstract改为摘要
%\renewcommand{\refname}{参考文献}   % 将References改为参考文献
%\renewcommand{\indexname}{索引}
%\renewcommand{\figurename}{图}
%\renewcommand{\tablename}{表}
%\renewcommand{\appendixname}{附录}

\setcounter{secnumdepth}{3} %设置编号深度
%\renewcommand{\thepart}{\textbf{第\arabic{part}部分}}
\renewcommand{\thesection}{第\arabic{chapter}.\arabic{section}节}
\renewcommand{\thesubsection}{\arabic{chapter}.\arabic{section}.\arabic{subsection}}
\renewcommand{\thesubsubsection}{(\arabic{subsubsection}) }
%\renewcommand{\thefootnote}{\alph{footnote}}    %设置脚注样式为小写字母


%-------------------------------设置页面属性-----------------------------%
\usepackage{indentfirst}
\setlength{\parindent}{2em} %设置自动缩进为两格
%\renewcommand{\baselinestretch}{1.2}   %设置 1.2 倍行距
%\setlength{\baselineskip}{20pt}    %设置行距
\geometry{left=2.7cm, right=2.7cm, top=3.5cm, bottom=2.6cm} %设置页边距
\linespread{1.2}            %设置 1.2 倍行距
%\setlength{\parskip}{0.5\baselineskip}  %设置段间距
\captionsetup{font={small}}     %设置图表标题大小


%-------------------------------设置数学环境-----------------------------%
\setatomsep{2em}    %设置全局化学键长度

%-------------------------------设置快捷命令-----------------------------%
\newcommand{\cdegree}{$^{\circ}$C}  %摄氏度



%--------------------------------添加页面元素----------------------------%
\title{\sanhao 氯化聚氯乙烯热稳定和润滑体系研究}
\author{
    \wuhao 朱浩南,高材1314,2013012433\\
    \wuhao 指导教师:武德珍,教授
}
\date{April,2017}

%设置页眉页脚
\fancyhf{}
\pagestyle{fancy}
\lhead{\xiaowuhao \leftmark}    %显示章节标题
\chead{\xiaowuhao 北京化工大学毕业设计(论文)}
\fancyfoot[C]{\thepage}
%设置双下划线页眉
\makeatletter %双线页眉
\def\headrule{{\if@fancyplain\let\headrulewidth\plainheadrulewidth\fi%
\hrule\@height 1.0pt \@width\headwidth\vskip1pt%上面线为1pt粗
\hrule\@height 0.5pt\@width\headwidth  %下面0.5pt粗
\vskip-2\headrulewidth\vskip-1pt}      %两条线的距离1pt
\vspace{6mm}}     %双线与下面正文之间的垂直间距
\makeatother


%-------------------------------文档部分---------------------------------%
\begin{document}

\maketitle

\begin{abstract}
    \textbf{\xiaosihao 介绍了碳纤维增强塑料复合材料(CFRP)在国内外的研究进展。介绍并总结了生产CFRP的各种成型工艺的优缺点。碳纤维复合材料在各个领域的具体应用,以及碳纤维的性能优势和改性方法。}
\end{abstract}

\tableofcontents

\chapter{绪论}

\section{CPVC 简介及其基本性能}

\subsection{CPVC 简介}
氯化聚氯乙烯(CPVC),也称为过氯乙烯,是将聚氯乙烯(PVC)进一步氯化改性得到的产物。在氯化过程中,一般可将 $\omega_{Cl}$\footnote{$\omega_{Cl}$: Cl 的质量分数,下同} 从 PVC 的 56.7\% 提高到 CPVC 的 61.0\%$\sim$68.0\%。研究表明,当氯含增加至 65\% 以上时,CPVC 的拉伸强度和弯曲强度呈直线上升,同时脆性也随之增大。随着氯含量的增加,共价键极性增大,分子间相互作用力增强,使 CPVC 树脂在物理力学性能,特别是耐候性、抗老化性、耐化学腐蚀性、热变形温度、阻燃自熄性等均比 PVC 有较大的提高,是近几年来应用领域发展速度较快的新型塑料材料。

\subsection{CPVC 分子结构}
CPVC 是 PVC 与 \chemfig{Cl_2} 在热及引发剂等作用下反应生成的产物。在氯化反应中,氯原子优先进攻 PVC 分子链中的 \chemfig{-CH_2-} 基团,而不是 \chemfig{-CHCl-} 基团,因此得到的 CPVC 主要是由 1,2-二氯乙烯链节构成。当 CPVC 的 $\omega_{Cl}$ 低于 63\% 时,生成的大部分为 \chemfig{-CHCl-CHCl-} 结构,具有一定的偶极矩,使得分子间范德华力增强;当 $\omega_{Cl}$ 高于 63\% 时,才生成极性较小的 \chemfig{-CH_2-CCl_2-} 结构。由此可见,PVC 的氯化主要发生在亚甲基碳原子上,生成 1,2-二氯乙烯链节;其次再发生在次甲基碳原子上,生成 1,1-二氯乙烯链节。随着氯化程度的提高,1,1-二氯乙烯与 1,2-二氯乙烯链节含量的比值增大。最终 CPVC 的 $\omega_{Cl}$ 由通氯量决定,其性能主要取决于氯化工艺。
    
\subsection{氯在 CPVC 中的分布}
CPVC 的结构单元主要包括 3 种基本结构(见图\ref{fig1})。其中各种结构单元在分子链中的含量与分布情况会在很大程度上影响分子链的断裂速率和方式,从而对 CPVC 的热稳定性以及加工性能产生很大的影响。因此,测定 CPVC 中的 $\omega_{Cl}$ 及在分子链中的分布情况是非常重要的。

\begin{figure}[htbp]
    \begin{center}
        \begin{minipage}[t]{0.2\linewidth}
            \centering
            \chemfig{-C(-[2]Cl)(-[6]H)-C(-[2]H)(-[6]H)-}
        \end{minipage}
        \begin{minipage}[t]{0.2\linewidth}
            \centering
            \chemfig{-C(-[2]Cl)(-[6]H)-C(-[2]H)(-[6]Cl)-}
        \end{minipage}
        \begin{minipage}[t]{0.2\linewidth}
            \centering
            \chemfig{-C(-[2]Cl)(-[6]H)-C(-[2]Cl)(-[6]Cl)-}
        \end{minipage}
    \end{center}
    \caption{CPVC 分子链中 3 种基本结构单元}
    \label{fig1}
\end{figure}

\subsection{CPVC 相对于 PVC 的优缺点}
\begin{itemize}
    \item{
        优点:\par
        CPVC 的 $T_g$ 比 PVC 高 50$\sim$60\cdegree\footnote{$T_{g, CPVC}$ 来自实验数据},阻燃性能也有所提高。并且保持了 PVC 原有的优点,即具有良好的耐化学腐蚀性、电绝缘性、耐候性等。CPVC 在沸水中不变形,是应用前景广阔的耐热耐腐蚀塑料材料。
    }
    \item{
        缺点:\par
        \begin{enumerate}[(1) ]
            \item CPVC 树脂的熔融温度与热分解温度相近,可加工温度范围小(180$\sim$190\cdegree),容易发生热分解;
            \item CPVC 熔体黏度高,约为 PVC 树脂熔体黏度的 3 倍左右,加工成型能耗大;
            \item 制品脆性大,冲击强度较低。
        \end{enumerate}
    }
\end{itemize}

\subsection{CPVC 性能特点}
CPVC 树脂在塑料管材(冷热水管、化工管、电力电缆护套、喷灌水管等)方面应用广泛,主要得益于其具有如下的优良特性。

\begin{enumerate}[(1) ]
    \item 与其他塑料管材相比,CPVC 树脂具有拉伸强度高、热膨胀系数小、热传导率低、难燃、氧气透过率小等特点,具体数据见表 \ref{tab1}。
    
    \begin{table}[htbp]
        \caption{CPVC 管材与其他塑料管材主要力学性能对比\cite{9}}
        \label{tab1}
        \begin{center}
        \scriptsize{
            \begin{tabular}{cccccc}
                \hline
                塑料管材 & \makecell[c]{拉伸强度 \\ (23\cdegree)/MPa} & \makecell[c]{热膨胀系数 \\ $\rm{\times 10^4/K^{-1}}$} & \makecell[c]{热传导率/ \\ $[\rm{W/(m\cdot K)}]$} & \makecell[c]{氧指数/ \\ \%} & \makecell[c]{氧气透过量(70\cdegree、1个大气压)/ \\ $[\rm{cm^3/(m^2\cdot d)}]$}  \\
                \hline
                CPVC 管材 & 55 & 0.7 & 0.14 & 60 & <1 \\
                PVC 管材 & 50 & 0.7 & 0.14 & 45 & <1  \\
                PP-R 管材 & 30 & 1.5 & 0.22 & 18 & 13$\sim$16 \\
                PE-X 管材 & 25 & 1.5 & 0.22 & 17 & 13 \\
                PB 管材 & 27 & 1.3 & 0.22 & 18 & 16   \\
                \hline
            \end{tabular}
        }
        \end{center}
    \end{table}
    
    \item 耐化学腐蚀性能好。工业用化学药剂大都会腐蚀金属设备,导致渗漏、流程限制、使用寿命短等问题。CPVC 不仅在常温下耐化学腐蚀性能优异,而且在较高温度下,CPVC 仍能保持较好的耐酸、耐碱、耐腐蚀性能,远优于 PVC 以及其他树脂。CPVC 在许多应用方面可取代传统材料,用以应对需要直接接触腐蚀性物品的场合,如处理质量较差的水、酸性物质、碱性物质以及其他具有腐蚀性的溶液。CPVC 能提供较长的使用寿命、低维修成本,并拥有良好的环境适应力。
    \item 阻燃性能好。由于氯含量的增加,CPVC 拥有优于 PVC 的阻燃性能。CPVC 塑料在空气中能够发生自熄,同时具有限制火焰扩散及低烟雾生成量等特性。
\end{enumerate}

\subsection{CPVC 降解机理}
PVC 在加工过程中不太稳定,极易发生降解脱 HCl,其重要原因是 PVC 分子链中存在着多种结构缺陷。通过红外光谱(IR)及核磁共振波谱观察发现,PVC 中的异常结构主要包括头-头结构、不饱和双键结构(末端双键、内部双键及共扼双键)、不稳定氯结构(烯丙基氯与叔氯)、支链结构(短支链结构与长支链结构)及二氯末端结构等。不稳定氯结构包括烯丙基氯和叔氯结构,其中烯丙基氯结构的含量远远高于叔氯结构的含量,极易诱发 PVC 脱 HCl。CPVC 与 PVC 结构相似,分子链中也存在着这些结构缺陷。研究表明:CPVC 树脂的加工稳定性远不如 PVC。CPVC 在加工过程中,明显存在着热分解现象,分解放出 HCl 气体后树脂严重变色。CPVC 发生热降解时,分子链中所存在的叔碳原子或末端双键会使 CPVC 的降解加剧。\par
CPVC 中氯原子沿碳链分布复杂,CPVC 的化学结构相当于氯乙烯、1,2-二氯乙烯以及1,1-二氯乙烯的三元共聚物。CPVC 分子中主要结构的含量为:\chemfig{-CHCl-} 65\%$\sim$70\%(摩尔分数,下同);\chemfig{-CH_2-} 20\%$\sim$30\%;\chemfig{-CCl_2-} 5\%$\sim$10\%。随着氯含量的增加,\chemfig{-CHCl-} 和 \chemfig{-CCl_2-} 两种结构单元的总量增加,\chemfig{-CH_2-} 结构单元减少。在氯含量大于 65\%(质量分数)以后,CPVC 分子的主要性能由\chemfig{-CHCl-CHCl-} 结构控制,随着 \chemfig{-CHCl-CHCl-} 结构的增加,CPVC 的玻璃化转变温度提高,耐热性增强。提高 CPVC 的性能需要在增加 \chemfig{-CHCl-} 结构和减少 \chemfig{-CH_2-} 结构的同时尽量避免 \chemfig{-CCl_2-} 结构结构和各种缺陷结构的产生。\chemfig{-CCl_2-} 结构会使分子链的极性减小,导致材料的玻璃化转变温度相应减小;另外,\chemfig{-CCl_2-} 结构易使材料受热脱 HCl,使分子链容易受热分解,热稳定性变差。\par
CPVC 分解的机理主要包括自由基机理、离子-分子机理和单分子机理。其中自由机理最为普遍,已成为稳定剂形成和发展的理论基础。

\begin{enumerate}[(1) ]
    \item 聚氯乙烯分子中某些薄弱结构,特别是烯丙基氯分解,产生氯自由基。
        \begin{center}
        \schemestart
            \chemfig{CH=CH-CH(-[2]Cl)-CH_2}
            \arrow(.mid east--.mid west)
            \chemfig{CH=CH-\lewis{2.,C}H-CH_2}
            \+
            \lewis{0.,Cl}
        \schemestop
        \end{center}
    \item 氯自由基从聚氯乙烯分子中吸取氢原子,形成链自由基。\par
        \begin{center}
        \schemestart
            \chemfig{\lewis{6.,C}H-CH(-[2]{Cl})-CH_2-CH(-[2]{Cl})}
            \arrow(.mid east--.mid west)
            \chemfig{CH=CH-CH_2-CH(-[2]{Cl})}
            \+
            \lewis{0.,Cl}
        \schemestop
        \end{center}
    \item 聚氯乙烯链自由基脱出 Cl 自由基,在大分子中形成双键。新生成的 Cl 自由基使两步反应反复进行,即发生所谓的拉链式脱 HCl 反应。\par
        \begin{center}
        \schemestart
            \chemfig{CH_2CHClCH_2CHClCH_2CHCl}
            \arrow(.mid east--.mid west)
            \chemfig{CH_2-\chemabove{C}{+}H-\chemabove{C}{-}H-CHClCH_2CHCl}
        \schemestop
        \end{center}
\end{enumerate}

\subsection{CPVC 配方设计、混料}
不同用途和性能的 CPVC 制品其配方设计不同,但是其基本配方都含有热稳定剂、润滑剂及其他助剂(如加工改性剂 、冲击改性剂 、填料、 光稳定剂、着色剂 、抗静电剂等)。\par
热稳定剂是 CPVC 加工中必不可少的助剂之一。润滑剂的作用在于减小物料和设备之间的摩擦力(外润滑作用),以及物料分子链之间的内摩擦力(内润滑作用)。\par
加入到 CPVC 树脂中的助剂,必须要经过混料,依靠混料设备的搅拌、振动、翻滚、研磨达到扩散、对流和剪切的作用,使物料混合均匀,保证后续成型的一致性。

\section{CPVC 的合成方法与结构}

\subsection{氯化反应机理}
实验室中主要采用气固相光催化法氯化 PVC 制备 CPVC,此反应为自由基反应,反应过程分为链引发、链传递、链终止三个步骤,反应机理见:\par

链引发反应:
    \begin{center}
        \schemestart
            \chemfig{Cl_2}
            \arrow(.mid east--.mid west)
            2\lewis{0.,Cl}
        \schemestop
    \end{center}

链传递反应:
    \begin{center}
        \schemestart
            \lewis{0.,Cl}
            \+
            \chemfig{-CHClCH_2-}
            \arrow(.mid east--.mid west)
            \chemfig{-CHClCHCl-}
            \+
            \lewis{0.,H}
        \schemestop
    \end{center}

    \begin{center}
        \schemestart
            \lewis{0.,Cl}
            \+
            \chemfig{-CHClCHCl-}
            \arrow(.mid east--.mid west)
            \chemfig{-CHClCCl_2-}
            \+
            \lewis{0.,H}
        \schemestop
    \end{center}
    
    \begin{center}
        \schemestart
            \lewis{0.,H}
            \+
            \chemfig{Cl_2}
            \arrow(.mid east--.mid west)
            \chemfig{HCl}
            \+
            \lewis{0.,Cl}
        \schemestop
    \end{center}

链终止反应:
    \begin{center}
        \schemestart
            2\lewis{0.,Cl}
            \arrow(.mid east--.mid west)
            \chemfig{Cl_2}
        \schemestop
    \end{center}

常见的引发方式主要有单纯热引发、紫外光引发及低温等离子体引发。单纯热引发方式即单纯依靠加热使 PVC 分子产生自由基从而制备 CPVC,所得产品的 $\omega_{Cl}$ 较低,反应过程中

\subsection{CPVC 合成方法}
目前,CPVC 树脂的生产工艺按氯化介质不同分为溶剂法、水相悬浮法和气固相法。溶剂法由于使用有机溶剂、能耗较高,目前几乎被淘汰。水相悬浮法具有操作简单、产品性能较好等优点,是目前国内外 CPVC 生产所采用的主要方法。但该法流程较长,生产“三废”较多,成本相对较高。用气固相搅拌式氯化法生产 CPVC,流程简单、污染物排放小,但传热效果较差,不适宜大规模生产。\par
其中一种 CPVC 制备方法:准确称取 5.0 g 的 PVC 粉末,装入流化床反应器中。反应器采用金属镀膜加热。料温达到 50$\sim$70\cdegree 时,通入 \chemfig{N_2} 防止 PVC 被氧化。温度升至 80\cdegree 后,开始加大 \chemfig{N_2} 流量使物料流化,并保持稳定。待达到氯化温度后,打开紫外灯,并开始通 \chemfig{Cl_2} 通过调节 \chemfig{N_2} 与 \chemfig{Cl_2} 流量改变原料气中的 $\varphi_{Cl_2}$,尾气用碱液吸收。反应完成后,取出产品,用蒸馏水浸泡 0.5 h,抽滤,重复操作至中性后于 60\cdegree 真空干燥至恒重。称量 CPVC 的质量,并分析产品中 $\omega_{Cl}$。为了充分活化 \chemfig{Cl_2} 并防止紫外光能量过高时造成 PVC 分解,选择了能量相对适中的波长为 300 nm 的紫外光作为引发光源。

\section{CPVC 树脂的应用}
CPVC 具有卓越的耐高温、抗腐蚀和阻燃性能,因此其市场应用状况良好。自 1960 年开始,CPVC 管材在美国开始应用,目前在北美已被普遍使用。其市场占有率由 1995 年的 20\% 增加至 2000 年的 30\%,2000 年的总销售量比 1984 年高 3 倍。\par
近 20 年来,我国的 CPVC 管材也高速发展,CPVC 树脂已成为除普通 PVC 树脂外用量最大的含氯树脂品种。下面介绍几种 CPVC 树脂在国内的应用情况。


\section{CPVC 热稳定剂概述}

热稳定剂是一类能防止或减少聚合物在加工使用过程中受热而发生降解或交联,延长复合材料使用寿命的添加剂。CPVC 常用的热稳定剂主要种类有:铅盐类热稳定剂、金属皂复合热稳定剂、有机锡热稳定剂、稀土类热稳定剂。\par

\subsection{热稳定剂分类}
\begin{enumerate}[(1) ]
    \item 最重要的铅盐类稳定剂有三碱式硫酸铅(\chemfig{3PbO·PbSO_4·H_2O})、二碱式亚磷酸铅(\chemfig{2PbO·PbHPO_3})、二碱式硬脂酸铅(\chemfig{2PbO·Pb{(C_{17}H_{35}COO)}_2})和铅白(\chemfig{2PbCO_3·Pb{(OH)}_2})。铅盐稳定剂的热稳定作用较强,具有良好的介电性能,且价格低廉,与润滑剂配比合理时可使 CPVC 树脂的加工温度范围变宽、加工及后加工的产品质量稳定,故应用广泛。但铅盐有毒,不能用于接触食品的制品,也不能制得透明的制品,而且易被硫化物污染生成黑色的硫化铅。稳定机理:铅元素具有优异的的吸收 HCl 能力,且生成的氯化铅不会对 CPVC 分解产生催化作用。
    \item 金属皂类稳定剂即高级脂肪酸金属盐,有铅、钡、钙、铬、锌盐等。这类稳定剂热稳定性一般,但透明性、润滑性较铅盐好。金属皂类稳定剂的性能随金属种类和酸根离子不同而变化,基本规律是铬锌皂初期热稳定性好,钡、钙、镁、铝长期热稳定性好。故一般多使用 Ca/Zn 复合稳定剂和 Ba/Zn 复合热稳定剂。稳定机理:Cd 皂和 Zn 皂能吸收 HCl 且能置换烯丙基氯抑制多链烯的生成。Ba 皂和 Ca 皂能吸收 HCl。复合稳定剂能抑制锌烧,提高稳定性能。
    \item 有机锡热稳定剂是含碳-锡的烷基化锡衍生物,烷基一般为甲基、丁基、辛基。它分为月桂酸锡、马来酸锡、硫醇锡。加入后制品的透明度好,耐候性优越。有机锡稳定剂制品与含硫物质接触会污染环境。辛基锡稳定剂和甲基锡稳定剂无毒,但使用价格昂贵。稳定机理:能置换不稳定氯原子和基团、与双键加成来起到稳定作用。
    \item 稀土类稳定剂稳定效果好且无毒,同时与其他稳定剂有协同作用,但近年来国家保护稀土元素,这也限制了其进一步发展。
\end{enumerate}

\subsection{热稳定机理}
\begin{enumerate}[(1) ]
    \item 吸收中和 HCl,抑制其自动催化作用。这类稳定剂包括铅盐类、有机酸金属皂类、有机锡化合物、环氧化合物、酚盐及金属硫醇盐等。它们可与 HCl 反应,抑制 CPVC 脱 HCl 的反应。
    \item 置换 CPVC 分子中不稳定的烯丙基氯原子抑制脱 HCl。如有机锡稳定剂与 CPVC 分子的不稳定氯原子发生配位结合,在配位体中,有机锡与不稳定氯原子置换。
    \item 与多烯结构发生加成反应,破坏大共轭体系的形成,减少着色。不饱和酸的盐或酯含有双键,与 CPVC 分子共轭双键发生双烯加成反应,从而破坏其共轭结构,抑制变色。
    \item 捕捉自由基,阻止氧化反应。如加入酚类热稳定剂能阻滞脱 HCl,是由于酚给出的H原子自由基能与降解的 CPVC 大分子自由基偶合,形成不能与 \chemfig{O_2} 反应的物质,而具有热稳定作用。这种热稳定剂可具有一种或兼具几种作用。
\end{enumerate}

\chapter{CPVC 润滑体系研究}

\chapter{CPVC 热稳定体系研究}

\bibliographystyle{plain}
\bibliography{biblio.bib}

\end{document}
