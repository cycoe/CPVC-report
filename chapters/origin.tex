\chapter{实验数据与处理}

\section{内外润滑剂组合对 CPVC 性能的影响}

\subsection{配方设计}
由上一组实验我们得到 PEW-0380 外润滑剂具有最好的综合润滑性能,因此在本组实验中我们选用 PEW-0380/G-60 润滑剂组合作为对照组,并且使用 A 蜡和 OP 蜡两种外润滑剂以及 OA2 蜡和 E 蜡两种内润滑剂两两组合配方,共 5 组配方做为实验组进行热稳定性和力学性能的测试。探究内外润滑剂对于 CPVC 热稳定性和加工性能的协同影响。具体配方如表 \ref{tab2Pre} 所示。

\begin{table}[!htb]
    \caption{CPVC 内外润滑剂组合配方设计表}
    \label{tab2Pre}
    \begin{center}
    \footnotesize{
        \begin{tabular}{cccccccccccc}
            \Xhline{1pt}
            \multirow{2}{*}{\makecell[c]{组别}} & \multirow{2}{*}{\makecell[c]{CPVC}} & \multirow{2}{*}{\makecell[c]{抗冲击\\改性剂}} & \multirow{2}{*}{\makecell[c]{有机锡}} & \multicolumn{3}{c}{外润滑剂} & \multicolumn{3}{c}{内润滑剂} & \multirow{2}{*}{\makecell[c]{加工\\助剂}} & \multirow{2}{*}{\makecell[c]{钛白粉}}   \\
            \cline{5-10}
            &&&& \makecell[c]{PEW-0380} & A 蜡 &  OP 蜡 & G-60 & OA2 蜡 & E 蜡	\\ 
            \Xhline{0.5pt}
            $S_1$ & 100 & 8 & 2 & 1.3 & & & 1.2 & & & 3 & 2	\\
            $S_2$ & 100 & 8 & 2 & & 1.3 & & & 1.2 & & 3 & 2	\\
            $S_3$ & 100 & 8 & 2 & & 1.3 & & & & 1.2 & 3 & 2	\\
            $S_4$ & 100 & 8 & 2 & & & 1.3 & & 1.2 & & 3 & 2	\\
            $S_5$ & 100 & 8 & 2 & & & 1.3 & & & 1.2 & 3 & 2	\\
            \Xhline{1pt}
        \end{tabular}
    }
    \end{center}
\end{table}

\subsection{玻璃化转变温度}

根据 $S_1 \sim S_5$ 五组配方的 损耗因子-时间 数据制得图 \ref{fig2Tg},取损耗因子的峰值温度作为试样的玻璃化转变温度。

\begin{figure}[!htb]
    \begin{center}
        \input{src/origin/2/tg.tex}
    \end{center}
    \caption{外润滑剂玻璃化转变温度}
    \label{fig2Tg}
\end{figure}

\section{热稳定体系测试}

\subsection{动态热稳定性}


\subsection{玻璃化转变温度}

