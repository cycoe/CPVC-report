\chapter{绪论}

\section{CPVC 简介及其基本性能}

\subsection{CPVC 简介}
氯化聚氯乙烯(CPVC),也称为过氯乙烯,是通过将聚氯乙烯(PVC)进一步氯化改性得到的产品。CPVC 最早由德国 \textit{I.G. Farben AG} 公司以溶液法制得。在 20 世纪 60 年代 初期,美国 \textit{Genova} 产品公司首次为冷热水分配系统制造了第一套 CPVC 管道和配件。而后,\textit{Genova} 与 CPVC 树脂的开发商 \textit{B.F. Goodrich} 公司合作开发了第一代用于 CPVC 黏合剂的四氢呋喃(THF)/甲基乙基酮(MEK)配方。我国在 1964 年由锦西化工研究院研制成功,在锦西化工总厂投入生产。\par
理想的聚 1, 2-二氯乙烯的 $\omega_{Cl}$\footnote{$\omega_{Cl}$: Cl 的质量分数} 为 73.7\%。在氯化过程中,一般可将 $\omega_{Cl}$ 从 56.7\%\footnote{$\omega_{Cl, PVC} = 56.7\%$} 提高到 61.0\%$\sim$68.0\%。研究表明,当氯含量达到 65\% 以上时,CPVC 的拉伸强度和弯曲强度呈线性增加,同时脆性也随之增大。由于分子在结构上的不规整性增大,分子结晶度下降,分子链的极性增强,因而其热变形温度大大上升\cite{14}。随着氯含量的增加,CPVC 分子中共价键极性增大,分子间相互作用力增强,使得 CPVC 树脂的物理力学性能,特别是耐候性、抗老化性、耐化学腐蚀性、热变形温度、阻燃自熄性等均比 PVC 有较大的提高,使其在塑料、建材、电气、医学、农业、橡胶、油漆、颜料、轮船、造纸、纺织、包装、涂料、钢材等方面有广泛的应用\cite{19}。

\subsection{CPVC 相对于 PVC 的优缺点}
\begin{itemize}
    \item{
        优点:\par
        CPVC 的 $T_g$ 比 PVC 高 20$\sim$30\cd\footnote{$T_{g, CPVC} = 106\sim 115$\cd, $T_{g, PVC} = 82$\cd},阻燃性能也有所提高。并且保持了 PVC 原有的优点,即具有良好的耐化学腐蚀性、电绝缘性、耐候性等。CPVC 在沸水中不变形,是应用前景广阔的耐热耐腐蚀塑料材料。
    }
    \item{
        缺点:\par
        \begin{enumerate}[(1) ]
            \item CPVC 树脂的熔融温度与热分解温度相近,可加工温度范围小\footnote{一般为 180$\sim$190\cd},容易发生热分解;
            \item CPVC 熔体黏度高,约为 PVC 树脂熔体黏度的 3 倍左右,加工成型能耗大;
            \item 制品脆性大,冲击强度较低。
        \end{enumerate}
    }
\end{itemize}

\subsection{CPVC 性能特点}
CPVC 树脂在塑料管材(冷热水管、化工管、电力电缆护套、喷灌水管等)方面应用广泛,主要得益于其具有如下的优良特性。

\begin{enumerate}[(1) ]
    \item CPVC 具有优异的力学性能与热学性能,具体数据见表 \ref{tabCPVCMach} 和表 \ref{tabCPVCTher}。
    
    \begin{table}[!htbp]
        \caption{通用 CPVC 的力学性能数据表}
        \label{tabCPVCMach}
        \begin{center}
        \footnotesize{
            \begin{tabular}{cc|cccc}
                \Xhline{1pt}
                \multicolumn{2}{c|}{物理参数} & \multicolumn{4}{c}{力学参数} \\
                \Xhline{1pt}
                \makecell[c]{密度/\\($\rm{g/cm^3}$)} & 吸水率 & \makecell[c]{杨氏模量($E$)/\\GPa} & \makecell[c]{拉伸强度($\sigma_t$)/\\MPa} & 断裂伸长率 & \makecell[c]{冲击强度/\\$\rm{kJ/m^2}$}    \\
                \Xhline{0.5pt}
                1.56 & 0.04$\sim$0.4 & 2.9$\sim$3.4 & 50$\sim$80 & 20$\sim$40\% & 2$\sim$5  \\
                \Xhline{1pt}
            \end{tabular}
        }
        \end{center}
    \end{table}
    
    \begin{table}[!htbp]
        \caption{通用 CPVC 的热学性能数据表}
        \label{tabCPVCTher}
        \begin{center}
        \footnotesize{
            \begin{tabular}{cccccc}
                \Xhline{1pt}
                \multicolumn{6}{c}{热学参数}    \\
                \Xhline{1pt}
                \makecell[c]{熔点($T_m$)/\\\cd} & \makecell[c]{玻璃化转变温度($T_g$)/\\\cd} & \makecell[c]{维卡软化点/\\\cd} & \makecell[c]{热导率/\\($\rm{W/(m\cdot K)}$)} & \makecell[c]{线膨胀系数($\alpha$)/\\K} & \makecell[c]{比热容($c$)/\\($\rm{kJ/(kg\cdot K)}$)} \\
                \Xhline{0.5pt}
                150 & 106$\sim$115 & 106$\sim$115 & 0.16 & $\rm{8 \times 10^{-5}}$ & 0.9    \\
                \Xhline{1pt}
            \end{tabular}
        }
        \end{center}
    \end{table}
    
    \item 与其他塑料管材相比,CPVC 树脂具有拉伸强度高、热膨胀系数小、热传导率低、难燃、氧气透过率小等特点,具体数据见表 \ref{tabCompare}。
    
    \begin{table}[!htbp]
        \caption{CPVC 管材与其他塑料管材主要力学性能对比\cite{9}}
        \label{tabCompare}
        \begin{center}
        \footnotesize{
            \begin{tabular}{cccccc}
                \Xhline{1pt}
                塑料管材 & \makecell[c]{拉伸强度 \\ (23\cd)/MPa} & \makecell[c]{热膨胀系数 \\ $\rm{\times 10^4/K^{-1}}$} & \makecell[c]{热传导率/ \\ $[\rm{W/(m\cdot K)}]$} & \makecell[c]{氧指数/ \\ \%} & \makecell[c]{氧气透过量(70\cd、1个大气压)/ \\ $[\rm{cm^3/(m^2\cdot d)}]$}  \\
                \Xhline{0.5pt}
                CPVC 管材 & 55 & 0.7 & 0.14 & 60 & <1 \\
                PVC 管材 & 50 & 0.7 & 0.14 & 45 & <1  \\
                PP-R 管材 & 30 & 1.5 & 0.22 & 18 & 13$\sim$16 \\
                PE-X 管材 & 25 & 1.5 & 0.22 & 17 & 13 \\
                PB 管材 & 27 & 1.3 & 0.22 & 18 & 16   \\
                \Xhline{1pt}
            \end{tabular}
        }
        \end{center}
    \end{table}
    
    \item 耐化学腐蚀性能好。工业用化学药剂大都会对金属设备造成腐蚀,导致渗漏、流程限制、使用寿命短等问题。CPVC 不仅在常温下耐化学腐蚀性能优异,而且在较高温度下,CPVC 仍能保持较好的耐酸、耐碱、耐腐蚀性能,远优于 PVC 以及其他树脂。CPVC 在许多应用方面可取代传统材料,用以应对需要直接接触腐蚀性物品的场合,如处理氨基磺酸、氯酸钠、硅酸钠、25\% 高锰酸钾、大于 25\% 浓度的丙二醇、酚、甲酸(< 25\% 浓度)、铬酸、丁酸(< 3\% 浓度)、氯胺、氯化铵等溶液。CPVC 能提供较长的使用寿命、较低的维修成本,并拥有良好的环境适应力。
    \item 阻燃性能好。CPVC 的氧指数为 60,所以其阻燃性高,燃烧后不产生滴落物,燃烧扩散慢,可限制烟雾的产生,并且不会产生有害气体。
    \item 很多聚烯烃材料(包括 PP、PE、PB 等)遇水中余氯时可能会发生分解,而 CPVC 则不会受水中的余氯的影响,不会出现裂痕和崩漏\cite{17, 18}。
\end{enumerate}

\subsection{CPVC 降解机理}
在加工过程中 PVC 具有较差的热稳定性,极易发生降解脱除 HCl,其重要原因是 PVC 分子链中的多种结构缺陷。通过红外光谱(IR)及核磁共振波谱观察发现,PVC 中的结构缺陷主要包括 头-头 结构、不饱和双键结构(末端双键、内部双键及共轭双键)、不稳定氯结构(烯丙基氯与叔碳氯)、支链结构(短支链结构与长支链结构)及二氯末端结构等。不稳定氯原子主要包括链端烯丙基氯、链内烯丙基氯和叔氯\cite{15},其中烯丙基氯结构的含量远远高于叔碳氯结构的含量,极易诱发 PVC 脱 HCl。CPVC 与 PVC 具有相似的结构,分子链中也存在着这些结构缺陷,并且CPVC 树脂的加工稳定性远不如 PVC\cite{6}。\par
\setatomsep{1.5em}
靖志国等 \cite{4} 利用 $\rm{^{13}}$C NMR 对 CPVC 分子链序列结构的测定发现,CPVC 分子中氯原子沿碳链分布情况复杂,其分子链结构相当于氯乙烯、1,2-二氯乙烯以及 1,1-二氯乙烯的三元共聚物。CPVC 分子中主要结构的摩尔分数为:\chemfig{\va{C}-CHCl-} 含量为 65\%$\sim$70\%;\chemfig{\va{C}-CH_2-} 含量为 20\%$\sim$30\%;\chemfig{\va{C}-CCl_2-} 含量为 5\%$\sim$10\%。随着 $\omega_{Cl}$ 的增大,\chemfig{\va{C}-CHCl-} 和 \chemfig{\va{C}-CCl_2-} 两种结构单元的总量增加,\chemfig{\va{C}-CH_2-} 结构单元减少。在 $\omega_{Cl}$ 大于 65\% 以后,CPVC 分子的主要性能由 \chemfig{\va{C}-[@{op,.75}]CHCl-CHCl-[@{cl,.25}]} \makebraces[3pt,3pt]{}{op}{cl} 结构控制,随着 \chemfig{\va{C}-[@{op,.75}]CHCl-CHCl-[@{cl,.25}]} \makebraces[3pt,3pt]{}{op}{cl} 结构的增加,CPVC 的玻璃化转变温度提高,耐热性增强。因而提高 CPVC 的性能需要在增加 \chemfig{\va{C}-CHCl-} 结构和减少 \chemfig{\va{C}-CH_2-} 结构的同时尽量避免 \chemfig{\va{C}-CCl_2-} 结构和各种缺陷结构的产生。\chemfig{\va{C}-CCl_2-} 结构会使分子链的极性减小,导致材料的玻璃化转变温度相应降低;另外,\chemfig{\va{C}-CCl_2-} 结构易使材料受热脱 HCl,使分子链容易受热分解,热稳定性变差。\par

研究表明,CPVC 的热分解分为两步进行\cite{12}。第一步为脱除 HCl,生成 \chemfig{-[,1.5,,,decorate,decoration=snake]C([2]-Cl)=C([2]-H)-[,1.5,,,decorate,decoration=snake]} 和 \chemfig{-[,1.5,,,decorate,decoration=snake]C([2]=O)-[,1.5,,,decorate,decoration=snake]} 结构单元以及它们的共轭结构。第二步按 Diels–Alder 机理发生缩合反应,进一步生成具有多环结构的含氯芳香族化合物。\par
含氯聚合物的脱 HCl 机理有单分子机理、离子型机理和自由基机理。马文光等\cite{22}通过 ESR\footnote{电子自旋法}的研究结果表明,CPVC 最可能发生的是自由基机理脱 HCl。其过程为不稳定氯原子在热的作用下脱离形成 \lewis{0.,Cl}$\;$,\lewis{0.,Cl}$\;$ 进一步引发拉链式分解反应。如反应 \eqref{eqCPVCDegrade1} 至反应 \eqref{eqCPVCDegrade3} 所示。

\setpolymerdelim[]
\setatomsep{2em}

氯化聚氯乙烯分子中某些薄弱结构,特别是烯丙基氯结构分解,产生 \lewis{0.,Cl}$\;$:
    \begin{equation}
    \small{
    \label{eqCPVCDegrade1}
    \schemestart
        \chemfig{\va{C}-[,1.2,,,decorate,decoration=snake]CH=CH-CH(-[2]Cl)-CH_2-[,1.2,,,decorate,decoration=snake]}
        \arrow(.mid east--.mid west)
        \chemfig{\va{C}-[,1.2,,,decorate,decoration=snake]CH=CH-\lewis{2.,C}H-CH_2-[,1.2,,,decorate,decoration=snake]}
        \makebraces[5pt,5pt]{}{op}{cl}
        \+
        \lewis{0.,Cl}
    \schemestop
    }
    \end{equation}

\lewis{0.,Cl}$\;$ 从氯化聚氯乙烯分子中吸取氢原子,形成链自由基。ESR 信号证明了大分子自由基的存在:
    \begin{equation}
    \small{
    \label{eqCPVCDegrade2}
    \schemestart
        \lewis{0.,Cl}
        \+
        \chemfig{\va{C}-[,1.2,,,decorate,decoration=snake]CH_2-CH([2]-Cl)-CH_2-CH([2]-Cl)-[,1.2,,,decorate,decoration=snake]}
        \arrow(.mid east--.mid west)
        \chemfig{\va{C}-[,1.2,,,decorate,decoration=snake]\lewis{2.,C}H-CH([2]-Cl)-CH_2-CH([2]-Cl)-[,1.2,,,decorate,decoration=snake]}
        \+
        \chemfig{HCl}
    \schemestop
    }
    \end{equation}

氯化聚氯乙烯链自由基脱除 \lewis{0.,Cl}$\;$,在大分子中形成双键。新生成的 \lewis{0.,Cl}$\;$ 促进反应 \eqref{eqCPVCDegrade2} 的发生,使两步反应反复进行,即发生拉链式脱 HCl 反应:
    \begin{equation}
    \small{
    \label{eqCPVCDegrade3}
    \schemestart
        \chemfig{\va{C}-[,1.2,,,decorate,decoration=snake]\lewis{6.,C}H-CH(-[2]{Cl})-CH_2-CH(-[2]{Cl})-[,1.2,,,decorate,decoration=snake]}
        \arrow(.mid east--.mid west)
        \chemfig{\va{C}-[,1.2,,,decorate,decoration=snake]CH=CH-CH_2-CH(-[2]{Cl})-[,1.2,,,decorate,decoration=snake]}
        \+
        \lewis{0.,Cl}
    \schemestop
    }
    \end{equation}

大分子末端的引发剂残基在热的作用下也会脱去形成自由基 \lewis{0.,R}$\;$,\lewis{0.,R}$\;$ 又引起进一步的链锁分解反应:
    \begin{equation}
    \small{
    \label{eqCPVCDegrade4}
    \schemestart
        \lewis{0.,R}
        \+
        \chemfig{\va{C}-[,1.2,,,decorate,decoration=snake]CH_2-CH(-[2]{Cl})-CH_2-CH(-[2]{Cl})-[,1.2,,,decorate,decoration=snake]}
        \arrow(.mid east--.mid west)
        \chemfig{\va{C}-[,1.2,,,decorate,decoration=snake]\lewis{2.,C}H-CH([2]-Cl)-CH_2-CH(-[2]{Cl})-[,1.2,,,decorate,decoration=snake]}
        \+
        \chemfig{RH}
    \schemestop
    }
    \end{equation}

在反应 \eqref{eqCPVCDegrade4} 之后,又会连续地发生反应 \eqref{eqCPVCDegrade3} 和反应 \eqref{eqCPVCDegrade2}。分解生成的大分子自由基也会发生链的转移,终止等反应形成支化、交联和不饱和双键结构。

\subsection{CPVC 不稳定氯原子的测定}
不稳定氯原子是引起 CPVC 热分解的主要原因。目前,测定不稳定氯原子含量的唯一有效的方法是酚烷基化法\cite{16}。烯丙基氯与叔碳氯较一般的氯原子有较大的反应活性,它们均可与苯酚发生取代反应,如图 \ref{phenol}所示。

\begin{figure}[!htb]
    \begin{center}
        \schemestart
        \chemfig{\va{C}-[,1.5,,,decorate,decoration=snake]C(-[2]H)(-[6]{Cl})-CH_2-C(-[2]{Cl})(-[6]{CH_2(-[6]CHCl([6]-[,1.5,,,decorate,decoration=snake]))})-CH_2-C(-[2]H)(-[6]{Cl})-[,1.5,,,decorate,decoration=snake]}
        \+
        \chemfig{*6([:-30]=-=(-[0]{OH})-=-)}
        \schemestop
    \end{center}
    \begin{center}
        \schemestart
        \arrow(.mid east--.mid west)
        \chemfig{\va{C}-[,1.5,,,decorate,decoration=snake]C(-[2]H)(-[6]{Cl})-CH_2-C([2]-*6(=-=(-[2]{OH})-=-))(-[6]{CH_2(-[6]CHCl([6]-[,1.5,,,decorate,decoration=snake]))})-CH_2-C(-[2]H)(-[6]{Cl})-[,1.5,,,decorate,decoration=snake]}
        \schemestop
    \end{center}
    \caption{利用 A. A. Caraculacu 酚烷基化法测定 CPVC 样品中不稳定氯原子含量的方法}
    \label{phenol}
\end{figure}


\section{CPVC 树脂的应用\cite{5}}
CPVC 具有卓越的耐高温、抗腐蚀和阻燃性能,而且与其他热塑性塑料相比,CPVC 成本更低。因此其被广泛应用于制造各种管材、板材、型材、片材、泡沫材料、防腐涂料等产品。自 20 世纪 60 年代开始,CPVC 管材开始在美国应用,目前在北美已被普遍使用。其市场占有率由 1995 年的 20\% 提高至 2000 年的 30\%,2000 年的总销售量比 1984 年高 3 倍。\par
近 20 年来,我国的 CPVC 树脂也高速发展,其已成为除普通 PVC 树脂外用量最大的含氯树脂品种。目前主要应用领域包括如下 6 个方面。

\subsection{管件、管材}
由于 CPVC 对严重腐蚀(如硫酸、铬酸、盐酸、烧碱、矿物盐、烃类有机物等化学药品腐蚀)和对恶劣环境侵蚀的优良耐受力,被广泛应用于冶炼工业、石油化工、造纸业、电镀业以及民用排污管道。
在民用方面,CPVC 管材广泛应用于家庭、办公室、医院、学校等楼房的采暖供热水管,亦可用作太阳能供水管和温泉供水管。使用 CPVC 冷热水管可提供一套清洁(细菌增长慢)、安全(静液压强度高)、易于安装(热膨胀系数低)、耐热、耐腐蚀、阻燃(氧指数高)、热损失少(热传导率低)的管道系统。

\subsection{阻燃材料}
CPVC 具有优异的阻燃性能及消烟性能,使其成为严格消防要求下的塑料产品的首选,可应用于电子电器产品的配件、包装材料、建筑材料、交通设施等领域,具有十分广阔的发展前景。

\subsection{涂料和黏合剂}
CPVC 在丙酮、氯代烃等有机溶剂中具有良好的溶解性,因此将 CPVC(或于其他树脂配合)与溶剂结合,可制得具有用途的涂料和黏合剂。CPVC 涂料可应用于化工防腐涂层、材料纤维制品的阻燃、建筑涂料等领域。CPVC 树脂具有优良的耐化学药品性、阻燃性以及耐热性,同时具有低的氧气透过量\footnote{表 \ref{tabCompare}},使其成为最重要的防腐涂料品种之一。CPVC 制得的黏合剂具有粘结强度高、耐化学性能好等优点,可用来黏合各种 PVC、CPVC 的管件。

\subsection{电力电缆用 CPVC 套管}
CPVC 套管主要用于电力电缆的铺设并起导向和保护作用,更多的用于路灯目埋地电缆套管,要求电缆套管具有优良的耐热性、绝缘性以及阻燃性。CPVC 套管的维卡软化温度高于 93\cd,不怕因电力传输过程中因焦耳效应产生的高温。


\section{CPVC 的结构与合成工艺}

\subsection{CPVC 分子结构}
\setatomsep{1.5em}
CPVC 是 PVC 与 \chemfig{Cl_2} 在热及引发剂等作用下反应生成的产物。研究表明,在氯化反应中,氯原子优先进攻 PVC 分子链中的 \chemfig{\va{C}-CH_2-} 基团,而不是 \chemfig{\va{C}-CHCl-} 基团,因此得到的 CPVC 主要是 \chemfig{\va{C}-[@{op,.75}]CHCl-CHCl-[@{cl,.25}]} \makebraces[3pt,3pt]{}{op}{cl} 链节构成。当 CPVC 的 $\omega_{Cl}$ 低于 63\% 时,生成的结构大部分为 \chemfig{\va{C}-[@{op,.75}]CHCl-CHCl-[@{cl,.25}]} \makebraces[3pt,3pt]{}{op}{cl},该结构具有一定的偶极矩,使得分子间范德华力增强;当 $\omega_{Cl}$ 高于 63\% 时,才逐渐生成极性较小的 \chemfig{\va{C}-[@{op,.75}]CCl_2-CH_2-[@{cl,.25}]} \makebraces[3pt,3pt]{}{op}{cl} 结构。由此可见,PVC 的氯化反应主要发生在亚甲基碳原子上,生成 \chemfig{\va{C}-[@{op,.75}]CHCl-CHCl-[@{cl,.25}]} \makebraces[3pt,3pt]{}{op}{cl} 链节;其次再发生在次甲基碳原子上,生成 \chemfig{\va{C}-[@{op,.75}]CCl_2-CH_2-[@{cl,.25}]} \makebraces[3pt,3pt]{}{op}{cl} 链节。随着氯化程度的提高,\chemfig{\va{C}-[@{op,.75}]CCl_2-CH_2-[@{cl,.25}]} \makebraces[3pt,3pt]{}{op}{cl} 与 \chemfig{\va{C}-[@{op,.75}]CHCl-CHCl-[@{cl,.25}]} \makebraces[3pt,3pt]{}{op}{cl} 链节含量的比值增大。最终 CPVC 的 $\omega_{Cl}$ 由通氯量决定,其性能主要取决于氯化工艺。
    
\subsection{氯在 CPVC 中的分布}
CPVC 的结构单元主要包括 3 种基本结构(见图\ref{fig1})。其中各种结构单元在分子链中的含量与分布情况会在很大程度上影响分子链的断裂速率和方式,从而对 CPVC 的热稳定性以及加工性能产生很大的影响。因此,测定 CPVC 中的 $\omega_{Cl}$ 及在分子链中的分布情况是非常重要的。
\setatomsep{2em}
\begin{figure}[!htbp]
    \begin{center}
        \begin{minipage}[t]{0.25\linewidth}
            \centering
            \chemfig{\va{C}-[,1.5,,,decorate,decoration=snake]C(-[2]Cl)(-[6]H)-C(-[2]H)(-[6]H)-[,1.5,,,decorate,decoration=snake]}
        \end{minipage}
        \begin{minipage}[t]{0.25\linewidth}
            \centering
            \chemfig{\va{C}-[,1.5,,,decorate,decoration=snake]C(-[2]Cl)(-[6]H)-C(-[2]H)(-[6]Cl)-[,1.5,,,decorate,decoration=snake]}
        \end{minipage}
        \begin{minipage}[t]{0.25\linewidth}
            \centering
            \chemfig{\va{C}-[,1.5,,,decorate,decoration=snake]C(-[2]Cl)(-[6]H)-C(-[2]Cl)(-[6]Cl)-[,1.5,,,decorate,decoration=snake]}
        \end{minipage}
    \end{center}
    \caption{CPVC 分子链中 3 种基本结构单元}
    \label{fig1}
\end{figure}

\subsection{氯化反应机理}
实验室一般采用气固相氯化法氯化 PVC 制备 CPVC,该反应为自由基机理,反应过程分为链引发、链传递、链终止三个阶段\cite{1},反应机理如反应 \eqref{eqChloride1}$\sim$\eqref{eqChloride5} 所示。\par

链引发反应:
    \begin{equation}
        \label{eqChloride1}
        \schemestart
            \chemfig{Cl_2}
            \arrow(.mid east--.mid west)
            2\lewis{0.,Cl}
        \schemestop
    \end{equation}

链传递反应:
    \begin{equation}
        \label{eqChloride2}
        \schemestart
            \lewis{0.,Cl}
            \+
            \chemfig{\va{C}-[,1.5,,,decorate,decoration=snake]C([2]-Cl)([6]-H)-C([2]-H)([6]-H)-[,1.5,,,decorate,decoration=snake]}
            \arrow(.mid east--.mid west)
            \chemfig{\va{C}-[,1.5,,,decorate,decoration=snake]C([2]-Cl)([6]-H)-C([2]-Cl)([6]-H)-[,1.5,,,decorate,decoration=snake]}
            \+
            \lewis{0.,H}
        \schemestop
    \end{equation}

    \begin{equation}
        \label{eqChloride3}
        \schemestart
            \lewis{0.,Cl}
            \+
            \chemfig{\va{C}-[,1.5,,,decorate,decoration=snake]C([2]-Cl)([6]-H)-C([2]-H)([6]-Cl)-[,1.5,,,decorate,decoration=snake]}
            \arrow(.mid east--.mid west)
            \chemfig{\va{C}-[,1.5,,,decorate,decoration=snake]C([2]-Cl)([6]-H)-C([2]-Cl)([6]-Cl)-[,1.5,,,decorate,decoration=snake]}
            \+
            \lewis{0.,H}
        \schemestop
    \end{equation}
    
    \begin{equation}
        \label{eqChloride4}
        \schemestart
            \lewis{0.,H}
            \+
            \chemfig{Cl_2}
            \arrow(.mid east--.mid west)
            \chemfig{HCl}
            \+
            \lewis{0.,Cl}
        \schemestop
    \end{equation}

链终止反应:
    \begin{equation}
        \label{eqChloride5}
        \schemestart
            2\lewis{0.,Cl}
            \arrow(.mid east--.mid west)
            \chemfig{Cl_2}
        \schemestop
    \end{equation}

常见的引发方式主要有单纯热引发、紫外光引发及低温等离子体引发。

\begin{enumerate}[(1) ]
    \item 单纯热引发方式即单纯依靠加热使 PVC 分子产生自由基从而制备 CPVC,所得产品的 $\omega_{Cl}$ 较低,反应过程中物料极易发黏变黄从而影响氯化反应的进行。
    \item 低温等离子体引发 PVC 氯化虽然能得到氯化均匀且具有较高 $\omega_{Cl}$ 的 CPVC,但是该引发方式制备 CPVC 较难实现工业化。
    \item 采用紫外光引发方式能够得到氯化均匀且具有较高 $\omega_{Cl}$ 的 CPVC,若能解决工程问题,有望实现工业化,以期解决目前 CPVC 生产工艺中存在的环境污染、产品后处理繁琐等弊端。
\end{enumerate}

\subsection{CPVC 合成方法}
目前,CPVC 树脂的生产工艺按氯化介质不同分为溶剂法、水相悬浮法和气固相法。

\subsubsection{溶液法}
溶液法是 CPVC 生产最早采用的方法,工艺比较成熟。它是将疏松型 PVC 树脂用适当的溶剂进行处理,然后在 80$\sim$100\cd 下加入偶氮二乙腈,通入氯气发生氯化反应生成 CPVC 树脂。在溶液中加入沉淀剂甲醇可使 CPVC 沉淀,抽滤后酸洗、干燥得氯含量为 64\%$\sim$75\% 的白色 CPVC 粉末产品。该方法制得的 CPVC 产品氯分布均匀,具有良好的溶解性能,易溶于 THF、二氯乙烷、氯苯等有机溶剂,适合用做涂料和黏合剂等。但由于为无规均质产品,其热稳定性、耐热性和机械性能较差,不能用于管材等硬质制品。溶剂法由于其使用四氯化碳、氯己烷等有机溶剂,毒性大、污染严重、溶剂回收复杂,并且能耗较高,目前几乎被淘汰。

\subsubsection{水相悬浮法}
水相悬浮法是以水或盐酸水溶液为介质,在搅拌作用下使 PVC 树脂悬浮于反应体系中。通入氮气充分除氧后,向反应体系中加入所需催化剂。缓慢向体系中通入氯气使反应开始进行,体系温度和压力上升。随着反应的进行,体系压力下降,至压力恒定时反应结束。泄压,在氮气保护下进行两次过滤、洗涤操作,滤饼进入稳定化工序\cite{23}。该法制得的非均质 CPVC 含氯量高、氯化均匀、热稳定性好。水相悬浮法具有操作简单、产品性能较好等优点,是目前国内外 CPVC 生产所采用的主要方法。但由于流程较长,生产“三废”较多,成本相对较高。

\subsubsection{气固相法}
由张向京等\cite{1}报导的由气固相法制备 CPVC 的方法如下:

\begin{enumerate}[(1) ]
    \item 准确称量 5.0 g PVC 粉末置于流化床反应器中。
    \item 采用金属镀膜给反应器加热。料温达到 50$\sim$70\cd 时,通入 \chemfig{N_2} 防止 PVC 被氧化。持续升温至 80\cd 后,加大 \chemfig{N_2} 流量使物料流化,并保持料温稳定。
    \item 待达到氯化温度时,打开紫外灯\footnote{为充分活化 \chemfig{Cl_2} 并防止紫外光能量过高时造成 PVC 分解,选择能量相对适中的波长为 300 nm 的紫外光作为引发光源。} 并开始通入 \chemfig{Cl_2},通过调节 \chemfig{N_2} 与 \chemfig{Cl_2} 的流量来改变原料气中的 $\varphi_{Cl_2}$\footnote{$\varphi_{Cl_2}$: \chemfig{Cl_2} 的体积分数},尾气用 KOH 吸收。
    \item 反应结束后,用蒸馏水浸泡样品 0.5 h,抽滤,重复操作至中性后于 60\cd 真空干燥至恒重。
\end{enumerate}

用该法生产 CPVC,具有流程简单、污染物排放小的优点,但由于传热效果较差,不适宜大规模生产。


\section{CPVC 加工与改性}
不同用途和性能的 CPVC 制品其配方设计不同,但是其基本配方都含有热稳定剂、润滑剂及其他助剂(如加工改性剂、冲击改性剂、填料、光稳定剂、着色剂、抗静电剂等)。

\subsection{热稳定剂}
由于 CPVC 树脂中的氯含量更高,在加工过程中较 PVC 更易发生热分解释放氯化氢,因此需选用合适的热稳定剂来防止降解的发生。\par
马玫等\cite{6}研究了复合铅体系对加工稳定性的影响。研究结果表明:将三盐基硫酸铅与二盐基亚磷酸铅复合使用能提供比单独使用更好的稳定性,与使用有机铅盐相近。铅盐稳定剂复合使用,对 CPVC 确有协同效应,6 份的铅盐稳定剂可以满足 CPVC 的加工需求。对于共稳定剂,亚磷酸盐能与铅盐稳定剂产生协同效应,随着亚磷酸酯的用量增大,CPVC 的塑化温度和平衡扭矩显著减小,提高了塑化效果的同时也使其性能(维卡软化温度、冲击强度)得到了提高。\par
柯伟席等\cite{8}研究了二盐、三盐、有机锡稳定剂和复合铅盐类热稳定剂以及辅助稳定剂对 CPVC 热稳定性及加工性能的影响。研究结果表明:复合铅盐类热稳定剂的稳定效果最好,且使得 CPVC 更易于加工。对于辅助热稳定剂,发现 Pb-St 和 Ba-St 具有良好的长期热稳定性及润滑性,并且两者具有协同作用。随着辅助热稳定剂用量的增加,CPVC 的塑化时间呈先上升后下降的趋势,Pb-St 和 Ba-St 的用量为 0.7$\sim$ 份比较合适。

\subsection{润滑剂}
CPVC 树脂比 PVC 树脂具有更高的熔体黏度以及更差的热稳定性,因此需要在加工过程中加入一定量的润滑剂来降低熔体的黏度和改善熔体的金属剥离性,从而延长树脂的热稳定时间以及降低加工能耗。\par
毛季红\cite{2}研究了内外润滑剂的品种和用量对 CPVC 材料流变性能的影响。该研究表明:在相同用量下,微晶石蜡和酰胺蜡作为外润滑剂具有最长的塑化时间以及最低的平衡转矩,因此认为其能够产生相对较好的润滑效果。进一步使用微晶石蜡和酰胺蜡,控制组成不变的情况下改变用量,发现外润滑剂用量越多则 CPVC 树脂的塑化时间越长,且平衡转矩也逐步降低,说明须足够的润滑剂用量\footnote{从实验结果来看,外润滑剂用量须在 2.5$\sim$3.0 份之间}才能保证较好的挤出效果。对于内润滑剂的研究发现,多元醇酯和脂肪酸酯的润滑效果相近,同时增加内润滑剂的用量可以促进树脂塑化。但内润滑剂熔点一般较低,对 CPVC 的热稳定性有较大的影响,因此需控制其用量。

\subsection{抗冲改性剂}
随着 CPVC 中氯含量的增加,分子链的极性增强,形成的大分子链间的范德华作用增大,致使 CPVC 的脆性增大,制品的抗冲击性能变差。考虑到 CPVC 树脂的加工性能以及在运输和安装过程中受到的冲击的影响,在 CPVC 的加工过程中须添加不同种类和用量的抗冲改性剂,以提高塑化质量及增加 CPVC 制品的低温抗冲性和韧性。\par
目前在 CPVC 制品的生产中,普遍使用 CPE\footnote{氯化聚乙烯}、MBS\footnote{(甲基丙烯酸甲醋-丁二烯-苯乙烯)共聚树脂}、ACR\footnote{抗 冲型丙烯酸醋橡胶}、ABS\footnote{丙烯睛-丁二烯-苯乙烯} 对 CPVC 进行改性。作为常用的抗冲改性剂,MBS、ACR 都具有典型的“核-壳”结构,都可在低用量下明显改善 CPVC 制品的脆性以及加工性能\cite{24}。


\section{热稳定剂概述}

CPVC 的热稳定性差,加工过程中易分解放出 HCl,生成不饱和共轭多烯,导致制品变色、变硬和烧焦。热稳定剂是一类能防止或减少聚合物在加工使用过程中受热而发生降解或交联,延长复合材料使用寿命的添加剂。CPVC 常用的热稳定剂种类主要有铅盐类热稳定剂、金属皂复合热稳定剂、有机锡热稳定剂和稀土类热稳定剂。

\subsection{热稳定机理}
热稳定剂可以通过取代不稳定氯原子、中和 HCl、与不饱和结构发生反应等方式抑制 CPVC 分子的降解。吸收降解早期阶段释放出的 HCl,以防止内在自动催化反应的发生。热稳定剂的稳定机理主要有以下 4 种:
\begin{enumerate}[(1) ]
    \item 吸收中和 HCl,抑制其自动催化作用。这类稳定剂包括铅盐类、有机酸金属皂类、有机锡化合物、环氧化合物、酚盐及金属硫醇盐等。它们可与 HCl 反应,抑制 CPVC 脱 HCl 的反应。
    \item 置换 CPVC 分子中不稳定的烯丙基氯原子抑制脱 HCl。如有机锡稳定剂与 CPVC 分子的不稳定氯原子发生配位结合,在配位体中,有机锡与不稳定氯原子置换。
    \item 与多烯结构发生加成反应,破坏大共轭体系的形成,减少着色。不饱和酸的盐或酯含有双键,与 CPVC 分子共轭双键发生双烯加成反应,从而破坏其共轭结构,抑制变色。
    \item 捕捉自由基,阻止氧化反应。如加入酚类热稳定剂能阻滞脱 HCl,是由于酚给出的H原子自由基能与降解的 CPVC 大分子自由基偶合,形成不能与 \chemfig{O_2} 反应的物质,而具有热稳定作用。这种热稳定剂可具有一种或兼具几种作用。
\end{enumerate}

\subsection{热稳定剂分类}

\subsubsection{铅盐类热稳定剂}
铅盐类热稳定剂是开发最早的一类稳定剂,其生产成本低,热稳定性好。最重要的铅盐类稳定剂有三碱式硫酸铅(\chemfig{3PbO·PbSO_4·H_2O})、二碱式亚磷酸铅(\chemfig{2PbO·PbHPO_3})、二碱式硬脂酸铅(\chemfig{2PbO·Pb{(C_{17}H_{35}COO)}_2})和铅白(\chemfig{2PbCO_3·Pb{(OH)}_2})。铅盐稳定剂的热稳定作用较强,具有良好的介电性能,且价格相对低廉。与润滑剂配比合理时可使 CPVC 树脂的加工温度范围变宽,加工及后加工的产品质量稳定,故应用广泛。但铅盐有毒,不能用于接触食品的制品,也不能制得透明的制品,而且易被硫化物污染生成黑色的硫化铅。稳定机理:铅元素具有优异的的吸收 HCl 能力,且生成的氯化铅不会对 CPVC 分解产生催化作用。

\subsubsection{金属皂类热稳定剂}
金属皂是高级脂肪酸金属盐的总称,作为 CPVC 热稳定剂的金属皂中,金属基一般为Pb、Ba、Ca、Cd、Zn、Mg等。脂肪酸基一般为硬脂酸、油酸等,其中硬脂酸最为常用。依据稳定机理和功能的不同,金属皂稳定剂可分为两大类:一类是以 Cd 和 Zn 为金属基,称为主金属皂稳定剂,它们能够吸收 HCl 且能置换烯丙基氯抑制多链烯的生成。但其生成的金属氯化物是路易斯酸,能够促进脱 HCl 反应的进行;另一类以 Ba、Sr、Ca、Mg 等碱土金属为金属基,称为辅助金属皂稳定剂,其仅仅显示捕获 HCl 的作用,但生成的金属氯化物对脱 HCl 无催化作用,并能有效置换主金属稳定剂反应生成的氯化物。金属皂热稳定剂单独使用都无法达到理想的效果,因此一般将其配合使用,使其发挥协同效应,从而达到最好的热稳定作用。这类稳定剂热稳定性一般,但透明性、润滑性较铅盐好。

\subsubsection{有机锡热稳定剂}
有机锡是热稳定性能较好的 CPVC 热稳定剂之一,其透明性好且大多无毒。常用的有机锡稳定剂可分为含硫有机锡和有机锡羧酸盐。含硫有机锡主要为硫醇有机锡和有机锡硫化物,这类稳定剂与 Pb、Cd 皂并用时热稳定效果极好,且透明性好,但其会产生硫化污染。有机锡羧酸盐主要包括脂肪酸锡盐、月桂酸锡盐和马来酸锡盐,其热稳定性不如含硫有机锡。有机锡稳定剂热稳定性好但价格太高,限制了其广泛推广应用。

\subsubsection{稀土类热稳定剂}
稀土稳定剂是我国特有的稳定剂体系,与我国有丰富的稀土资源有关。稀土包括原子序号从 57$\sim$71 的 15 个镧系元素及其相近的钇、钍共 17 个元素。稀土稳定剂有促进 CPVC 塑化的特点,目前我国在管材、型材方面大力推广应用。稀土稳定剂主要包括稀土的氧化物、氢氧化物及其有机弱酸盐。稀土类稳定剂稳定效果好且无毒,同时与其他稳定剂有协同作用。

\subsection{钙锌复合热稳定剂}
待定

\subsection{辅助型热稳定剂}
辅助热稳定剂单独使用时稳定效果较差,但它作为辅助热稳定剂与主稳定剂配合使用,则能大大增强主稳定剂的热稳定效果。目前广泛使用的有机热稳定剂主要有亚磷酸酯、环氧化合物、多元醇、含氮化合物、含硫化合物、$\beta$-二酮化合物等\cite{26}。


\section{润滑剂概述\cite{2}}
润滑剂是一类用于降低熔体黏度和改善熔体的金属剥离性,从而延长材料的热稳定性以及提高加工性能的加工助剂。从实现的功能上进行分类,润滑剂可分为外润滑剂和内润滑剂。

\subsection{外润滑剂}
外润滑剂极性小,但却具有很长的碳链,其作用主要是降低聚合物和加工机械之间的摩擦,改善熔体的金属剥离性,调节混合物的熔点以及流变学性能,减少挤出负载,减少热稳定剂的消耗量。外润滑剂与聚合物的相容性较差,容易从熔料中往外迁移,在成型过程中能在熔料与模具间形成一层很薄的隔离膜,使塑料不粘住模具表面。但外润滑剂的加入对力学和耐热性能均会产生负面的影响。因此,在满足加工性能的基础上外润滑剂的假如落越少越好,即要求在相同的加入量时能够产生较好的润滑效果。

\subsection{内润滑剂}
内润滑剂与聚合物有良好的相容性,它在聚合物内部起着降低聚合物分子间内聚力的作用,从而降低大分子之间的摩擦,降低熔体黏度,降低熔体破裂和出膜膨胀。内润滑剂和聚合物长链分子间的结合是不强的,它们可能产生类似于滚动轴承的作用,因此其自身能在熔体流动方向上排列,从而互相滑动,使得内摩擦力降低。但加入内润滑剂对塑化时间以及产品的透明性的影响不大,但会降低材料的维卡软化点。

\subsection{润滑剂的分类}
润滑剂按化学结构可划分为脂肪酸酰胺类、烃类、脂肪酸类、酯类、醇类、金属皂类、复合润滑剂类。

\subsubsection{脂肪酸酰胺类润滑剂}

\begin{enumerate}[(1) ]
    \item 硬脂酸酰胺:白色或淡黄褐色粉末,相对密度 0.96,分子量 283,熔点 98$\sim$103\cd,溶于水,溶于热乙醇、氯仿、乙醚。具有优良的外部润滑效果和脱膜性,透明性、分散性、光泽性和电绝缘性亦佳,无毒。是 PVC、PS、UF 等树脂加工润滑剂,还可作为聚烯烃的爽滑剂和抗粘连剂。一般用量 0.1\%$\sim$2.0\%。
    \item N,N-亚乙基双硬脂酰胺(EBS):白色或乳白色粉末或粒状物。相对密度 0.98,分子量 593,熔点 142\cd,不溶于水,溶于热的氯代烃类和芳烃类溶剂。广泛用于爽滑剂、抗粘连剂、润滑剂和抗静电剂。无毒,适用于 PE、PP、PS、ABS 树脂及热固性塑料的内部和外部润滑剂。一般用量为 0.2\%$\sim$2.0\%。
    \item 油酸酰胺:白色粉末状、碎片状或珠粒状物。相对密度 0.90,分子量 281,熔点 68$\sim$79\cd,不溶于水,溶于乙醇等许多溶剂。无毒,可作为 PE、PP、PA 等塑料的爽滑剂、防黏剂,改善加工成型性能,还具有抗静电效果,可减少灰尘在制品表面的附着,在 PVC 加工成型中本品是良好的内部润滑剂。
    \item 芥酸酰胺:形状、性能及用途与油酸酰胺相似,比油酸酰胺更佳。
    \item 硬脂酸正丁酯(BS):淡黄色液体,相对密度 0.855$\sim$0.862,溶于大多数有机溶剂,微溶于甘油、乙二醇和某些胺类,与乙基纤维素相容,与硝酸纤维素、乙酸丁酸纤维素、氯化橡胶等部分相。本品无毒,作为树脂加工时的内部润滑剂,具有防水性和较好的热稳定性,可用于涂料。虽与 PVC 不相容,但可作为 PVC 透明片挤出、注塑、压延的润滑剂、脱膜剂。一般用量 0.5\%$\sim$1.0\%。
    \item 甘油三羟硬脂酸酯:粉末状物,熔点 85$\sim$87\cd。本品无毒,具有优良的耐热性和流动性。可作为 PVC、ABS、MBS 的润滑剂和爽滑剂和合成橡胶的脱膜剂。一般用量为 0.25\%$\sim$1.5\%。
\end{enumerate}

\subsubsection{烃类润滑剂}
\begin{enumerate}[(1) ]
    \item 微晶石蜡:白色或微黄色鳞片状或粒状物,固体相对密度 0.89$\sim$0.94,液体相对密度 0.78$\sim$0.81,熔点 70$\sim$90\cd,溶于非极性溶剂,不溶于极性溶剂。热稳定性、润滑性优于石蜡,但会降低凝胶化速度,故用量不宜过大。无毒,常与硬脂酸丁酯或高级脂肪酸并用,用于塑料润滑剂。一般用量 0.1\%$\sim$0.2\%。
    \item 液体石蜡:无色透明液体,相对密度 0.89,凝固点 -35$\sim$-15\cd,溶于苯、乙醚、二硫化碳,微溶于醇类,在热稳定及润滑性均良好。用于 PVC、PS 等树脂加工时,作为内润滑剂,与树脂相容性差。添加量一般为 0.3\%$\sim$0.5\%,过多时,反而使加工性能变坏。
    \item 固体石蜡:白色固体,相对密度 0.9,熔点 57$\sim$60\cd,不溶于水,溶于汽油、氯仿、二硫化碳、二甲苯、乙醚等有机溶剂,微溶于醇类。属于外润滑剂,可改善制品表面光泽,为非极性直链烃,不能润湿金属表面,也就是不能阻止 PVC 黏金属壁,只有与硬脂酸钙并用时,才能发挥协同效应,但其相容性、分散性和热稳定性均比较差。本品无毒,用于 PVC、PE、PP、PS、ABS、PBT、PET 及纤维素等塑料。
    \item 氯化石蜡:石蜡经氯化而制得。无臭透明液体,含氯量有 42\%,52\%,70\% 等多种,与 PVC 相容性好,还起增塑剂、阻燃剂的作用,但透明度差,用量在 0.3\% 以下,与其他增塑剂并用效果较好。一般用量 0.3\%。
    \item 聚乙烯蜡:又称低分子量聚乙烯,白色粉末或片状物,为乙烯的低度聚合产品。相对密度为 0.9$\sim$0.93,分子量 1000$\sim$ 5000,软化点 100$\sim$115\cd,具有良好的中期及后期润滑性,能起防黏剂作用,在色母粒加工中作颜料分散剂,在 PVC-U 中作润滑剂,在 PVC、PE、PP、ABS、PET、PBT 塑料成型中作润滑剂和脱模剂。一般用量 0.1\%$\sim$0.5\%。
    \item 氧化聚乙烯蜡:白色粉末或珠粒状固体,为含羧酸的低分子量聚乙烯,并含有醇、酮及酯类化合物,由于氧化使烷烃链上生成一定数量的羧基和烃基(均为极性基团,故提高了它在 PVC 的相容性,使其同时兼有良好的内、外润滑性能,并赋予制品良好的透明性和光泽性,与高级脂肪或脂肪酸进行部分酯化,或用氢氧化钙进行部分皂化,得到的衍生物均具优异的内、外润滑性能。主要应用于 PVC、PE、PP、ABS、PBT、PET 等树脂的优秀润滑剂。用量 0.1\%$\sim$1.0\%。
\end{enumerate}

\subsubsection{复合润滑剂}
复合润滑剂是具有良好的内、外润滑剂的功效。常用的复合润滑剂有:石蜡类、金属皂与石蜡复合、脂肪酰胺与其他润滑剂复合物、一褐煤蜡为主体的复合润滑剂、稳定剂与润滑剂的复合体系。

\subsubsection{硅氧烷润滑剂}
硅氧烷系作为脱模剂、防粘连剂和润滑剂广泛应用于酚醛、环氧、聚酯等塑料的加工成型上。常用的品种有聚硅氧烷、合成蜡、硅油、二氧化硅和硅藻土等。

\begin{enumerate}[(1) ]
    \item 甲基硅油:即聚二甲基硅氧烷,无色、无味,透明、黏稠液体,分子量为 5000$\sim$10000,溶于乙醚、苯、甲苯,部分溶于丙酮、乙醇、丁醇,不溶于甲醇、环已醇、石蜡油、植物油。可在 -50$\sim$200\cd 范围内使用。具有优良的耐高、低温性能,透光性、电性能、增水性和化学稳定性均良好。用作为脱模润滑剂。
    \item 苯甲基硅油:即聚甲基苯基硅氧烷,性能同甲基硅油。
    \item 乙基硅油:即聚二乙基硅氧烷,无色或浅黄色透明液体,平均分子量 300$\sim$10000。溶于乙醚、氯仿、甲苯。可与石油产品任意混合,使用温度 -70$\sim$150\cd,具有优良的润滑性和电绝缘性,表面张力较小,防水、耐化学腐蚀性能好。可以作为脱模剂和润滑剂应用于塑料、橡胶加工润滑剂。
\end{enumerate}

\subsection{常用树脂所适用的润滑剂}
\begin{enumerate}[(1) ]
    \item 聚氯乙烯:适用:液体石蜡、固体石蜡、高熔点石蜡、聚乙烯蜡、乙撑双硬脂酰胺、酯蜡硬脂酸丁酯、单硬脂酸甘油酯、金属皂、硬脂酸、硬脂醇。
    \item 聚乙烯、聚丙烯:适用:乙撑双硬脂酰胺、硬脂酰胺、油酸酰胺、硬脂酸钙、硬脂酸锌、高沸点石蜡、微晶石蜡、脂肪酸。
    \item 聚苯乙烯:适用硬脂酸锌、乙撑双硬脂酰胺、高熔点石蜡、硬脂酸丁酯。
    \item ABS 树脂:适用硬脂酸锌等金属皂、脂肪酰胺、乙撑双硬脂酰胺、高熔点石蜡。
    \item 聚酰胺:适用油酸酰胺、硬脂酰胺、乙撑双硬酯酰胺。
    \item PBT/PET 树脂:适用硬脂酸锌、硬脂酸钙、脂肪酰胺、高熔点石蜡、聚乙烯蜡。
    \item 酚醛、氨基树脂:适用硬脂酸锌等金属皂、脂肪酰胺、乙撑双硬脂硬酰胺、高熔点石蜡
\end{enumerate}
