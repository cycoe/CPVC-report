\chapter{结论}

\section{润滑体系}
本节实验对 AC-316、AC-617、AC-629 和 PEW-0380 四种外润滑剂进行了热性能和力学性能的测试。结果表明,AC-629 与 PEW-0380 外润滑剂为 CPVC 提供较好的润滑效果和塑化效果,使其热稳定性也略好于 AC-316 和 AC-617 组。但由于 AC-617、AC-629 的滴点较低,使其在加工过程中较早失效,使得加工能耗上升。在力学性能的测试中,4 组试样具有相近的拉伸强度与弯曲强度,但 PEW-0380 能使 CPVC 提供最好的冲击强度。因此在加工过程中,若加工温度不高,则外润滑剂可选用润滑效果较好的 AC-629 与 PEW-0380,若加工温度较高,则选用滴点较高的 AC-316 与 PEW-0380。

\section{热稳定体系}
在本节实验中,我们对 TMG-234、T-190A 和液体有机锡 3 种有机锡热稳定剂进行了热性能与力学性能的测试。实验结果表明,液体有机锡的短期热稳定性略好于 TMG-234 和 T-190A 热稳定剂,但液体有机锡的长期热稳定性较差且热稳定时间短。在塑化时间要求较短的情况下可选用液体有机锡热稳定剂,要求时间较长的情况下需选用 TMG-234、T-190A 热稳定剂。力学性能测试显示,液体有机锡使 CPVC 的强度与韧性的损失较小。同时 T-190A 体系的 $T_g$ 较高,其在较高温度下仍能使 CPVC 提供较高的强度与硬度。