%------------------氯化聚氯乙烯的热稳定和润滑体系研究-----------------%
%----------------------------朱浩南-----------------------------%

%---------------------------定义文档类型-------------------------%
%   \documentclass[param]{class}
%   param{
%       a4paper: A4纸
%       oneside: 单面打印
%       onecolumn: 单列排版
%       12pt: 默认字体大小
%   }
%   class{
%       report: 长报告、论文
%       article: 短文章
%       ctexrep: ctex report
%---------------------------------------------------------------%
    
\documentclass[a4paper, oneside, onecolumn, notitlepage, 12pt]{ctexrep}    %transmag 实现双列排版,摘要跨列

%----------------------------引入宏包---------------------------%
\usepackage[utf8]{inputenc} %特殊符号
%\usepackage{xeCJK}      %CJK引擎,使用 CTex 文档类后不再需要次宏包
%\usepackage{abstract}  %摘要与关键词
\usepackage{graphicx}   %插图工具包
\usepackage[super, square]{natbib}     %漂亮的bib引用    super-上标, square-方括号
\usepackage{bm}         %设置缩进
\usepackage{fancyhdr}   %定制页眉页脚
\usepackage{chemfig}    %化学方程式
\usepackage{amsmath}    %数学公式阵列
\usepackage{multirow}   %表格合并单元格
\usepackage{setspace}   %行距宏包
\usepackage{geometry}   %页边距宏包
\usepackage{fontspec}   %字体
\usepackage{enumerate}  %定制列表序号样式
\usepackage{makecell}   %表格内容换行
\usepackage{pifont}     %实现带圈数字,①开始于\ding{192}
\usepackage[perpage,symbol*]{footmisc}
\usepackage[justification=centering]{caption}   %标题设置
\usepackage{float}      %设定图片的位置    \[H]可固定图片位置
\usepackage{tikz}       %tikz图形库


%----------------------------设置字体族--------------------------%
%\CJKfamily{SimSun}
\setmainfont{Times New Roman}
\setsansfont{Arial}
\setmonofont{Noto Mono}
\setCJKmainfont[BoldFont=SimHei]{STSong}

%\newcommand{\song}{\CJKfamily{simsun}}  % 宋体
\newcommand{\hei}{\CJKfamily{SimHei}}   % 黑体


%------------------------------设置字体大小------------------------%  
\newcommand{\chuhao}{\fontsize{42pt}{\baselineskip}\selectfont}     %初号  
\newcommand{\xiaochuhao}{\fontsize{36pt}{\baselineskip}\selectfont} %小初号  
\newcommand{\yihao}{\fontsize{28pt}{\baselineskip}\selectfont}      %一号  
\newcommand{\erhao}{\fontsize{21pt}{\baselineskip}\selectfont}      %二号  
\newcommand{\xiaoerhao}{\fontsize{18pt}{\baselineskip}\selectfont}  %小二号  
\newcommand{\sanhao}{\fontsize{15.75pt}{\baselineskip}\selectfont}  %三号  
\newcommand{\sihao}{\fontsize{14pt}{\baselineskip}\selectfont}      %四号  
\newcommand{\xiaosihao}{\fontsize{12pt}{\baselineskip}\selectfont}  %小四号  
\newcommand{\wuhao}{\fontsize{10.5pt}{\baselineskip}\selectfont}    %五号  
\newcommand{\xiaowuhao}{\fontsize{9pt}{\baselineskip}\selectfont}   %小五号  
\newcommand{\liuhao}{\fontsize{7.875pt}{\baselineskip}\selectfont}  %六号  
\newcommand{\qihao}{\fontsize{5.25pt}{\baselineskip}\selectfont}    %七号

%------------------------------设置标题样式-------------------------------%
%\renewcommand{\contentsname}{目录}  % 将Contents改为目录
%\renewcommand{\abstractname}{摘要}  % 将Abstract改为摘要
%\renewcommand{\refname}{参考文献}   % 将References改为参考文献
%\renewcommand{\indexname}{索引}
%\renewcommand{\figurename}{图}
%\renewcommand{\tablename}{表}
%\renewcommand{\appendixname}{附录}

\setcounter{secnumdepth}{3} %设置编号深度
%\renewcommand{\thepart}{\textbf{第\arabic{part}部分}}
\renewcommand{\thesection}{第\arabic{chapter}. \arabic{section}节}
\renewcommand{\thesubsection}{\arabic{chapter}. \arabic{section}. \arabic{subsection}}
\renewcommand{\thesubsubsection}{\arabic{chapter}. \arabic{section}. \arabic{subsection}. \arabic{subsubsection}}


% 实现带圈脚注样式
\DefineFNsymbols{circled}{{\ding{182}}{\ding{183}}{\ding{184}}
{\ding{185}}{\ding{186}}{\ding{187}}{\ding{188}}{\ding{189}}{\ding{190}}{\ding{191}}}
\setfnsymbol{circled}

% 设置小写字母脚注样式
%\renewcommand{\thefootnote}{\alph{footnote}}


%-------------------------------设置页面属性-----------------------------%
\usepackage{indentfirst}
\setlength{\parindent}{2em} %设置自动缩进为两格
%\renewcommand{\baselinestretch}{1.2}   %设置 1.2 倍行距
%\setlength{\baselineskip}{20pt}    %设置行距
\linespread{1.2}            %设置 1.2 倍行距
\geometry{left=2.7cm, right=2.7cm, top=3.5cm, bottom=2.6cm} %设置页边距
\setlength{\itemsep}{0pt}   %设置列表元素间距
%\setlength{\parskip}{0}  %设置段间距
\captionsetup{font={small}}     %设置图表标题大小


%-------------------------------设置数学环境-----------------------------%
\setatomsep{2em}    %设置全局化学键长度

%设置高分子的括号
\newcommand\setpolymerdelim[2]{\def\delimleft{#1}\def\delimright{#2}}
\def\makebraces[#1,#2]#3#4#5{%
\edef\delimhalfdim{\the\dimexpr(#1+#2)/2}%
\edef\delimvshift{\the\dimexpr(#1-#2)/2}%
\chemmove{%
\node[at=(#4),yshift=(\delimvshift)]
{$\left\delimleft\vrule height\delimhalfdim depth\delimhalfdim
width0pt\right.$};%
\node[at=(#5),yshift=(\delimvshift)]
{$\left.\vrule height\delimhalfdim depth\delimhalfdim
width0pt\right\delimright_{\rlap{$\scriptstyle#3$}}$};}}
\setpolymerdelim[]

%   调用时使用类似以下的命令
%   \setpolymerdelim[]  设定括号样式,若为圆括号则()
%   \chemfig{-[@{op,.75}]CH=CH-\lewis{2.,C}H-CH_2-[@{cl,.25}]}  此处.75  .25两个参数分别为两个化学键在括号左边部分的比例
%   \makebraces[5pt,5pt]{}{op}{cl}  此处 5pt 为两个括号的大小

%引用 tikz 装饰库
\usetikzlibrary{decorations.pathmorphing}


%-------------------------------设置快捷命令-----------------------------%
\newcommand{\cd}{$^{\circ}$C}  %摄氏度
\newcommand{\va}{\vphantom}         %使第一个化学键居中



%--------------------------------添加页面元素----------------------------%
\title{\sanhao 氯化聚氯乙烯热稳定和润滑体系研究}
\author{
    \wuhao 朱浩南,高材1314,2013012433\\
    \wuhao 指导教师:武德珍,教授
}
\date{April,2017}

%设置页眉页脚
\fancyhf{}
\pagestyle{fancy}
% \lhead{\xiaowuhao \leftmark}    %显示章节标题
\chead{\xiaowuhao 北京化工大学毕业设计(论文)}
\fancyfoot[C]{\thepage}
%设置双下划线页眉
\makeatletter %双线页眉
\def\headrule{{\if@fancyplain\let\headrulewidth\plainheadrulewidth\fi%
\hrule\@height 1.0pt \@width\headwidth\vskip1pt%上面线为1pt粗
\hrule\@height 0.5pt\@width\headwidth  %下面0.5pt粗
\vskip-2\headrulewidth\vskip-1pt}      %两条线的距离1pt
\vspace{6mm}}     %双线与下面正文之间的垂直间距
\makeatother

%-------------------------------文档部分---------------------------------%
\begin{document}

\maketitle

\begin{abstract}
    \xiaosihao 氯化聚氯乙烯(CPVC)的热稳定性差,熔体黏度高,加工时易发生热分解脱氯化氢,因此需要在 CPVC 加工过程中加入热稳定剂和润滑剂。
\end{abstract}

%\renewcommand\abstractname{Abstract}
%\begin{abstract}
%    \xiaosihao CPVC is an unstable
%\end{abstract}

\tableofcontents

\chapter{绪论}
\pagenumbering{arabic}
\setcounter{page}{1}

\section{CPVC 树脂简介及其基本性能}

\subsection{CPVC 树脂简介}
氯化聚氯乙烯(CPVC),是一种由聚氯乙烯(PVC)氯化改性得到的树脂产品。CPVC 最早由德国 \textit{I.G. Farben AG} 公司采用溶液法制得。在 20 世纪 60 年代 初期,美国 \textit{Genova} 产品公司首次为冷热水分配系统制造了第一套 CPVC 管道和配件。而后,\textit{Genova} 与 CPVC 树脂的开发商 \textit{B.F. Goodrich} 公司合作开发了第一代用于 CPVC 黏合剂的四氢呋喃(THF)/甲基乙基酮(MEK)配方。1964 年,在我国锦西化工研究院成功研制出了 CPVC 的合成方法,并在锦西化工总厂投入生产。\par
理想的聚 1, 2-二氯乙烯的 $\omega_{Cl}$\footnote{$\omega_{Cl}$: Cl 的质量分数} 为 73.7\%。在 PVC 氯化过程中,一般 $\omega_{Cl}$ 可由 56.7\%\footnote{$\omega_{Cl, PVC} = 56.7\%$} 提高到 61.0\%$\sim$68.0\%。研究表明,当 $\omega_{Cl}$ 达到 65\% 以上时,CPVC 的拉伸强度和弯曲强度随氯含量的增加呈线性增加,同时脆性也符合此规律。由于 CPVC 分子中的氯含量增大,分子的等规度下降,使得 CPVC 分子链的极性增强,最终导致分子链的刚性和材料的热变形温度明显增大\cite{14}。并且随着 $\omega_{Cl}$ 的增大,CPVC 树脂的物理力学性能,特别是耐候性、抗老化性、耐化学腐蚀性、热变形温度、阻燃自熄性等均比 PVC 有较大的提高,使其在管件管材、电子电器设备配件、建筑装潢材料、化工领域等方面有广泛的应用\cite{19}。

\subsection{CPVC 性能特点}
CPVC 树脂在管件管材、电子电器设备配件等方面应用广泛,主要得益于其具有如下的优良特性。

\begin{enumerate}[(1) ]
    \item CPVC 具有优异的力学性能与热学性能,具体数据见表 \ref{tabCPVCMach} 和表 \ref{tabCPVCTher}。
    
    \begin{table}[!htb]
        \caption{通用 CPVC 的力学性能数据表}
        \label{tabCPVCMach}
        \begin{center}
        \footnotesize{
            \begin{tabular}{cc|cccc}
                \borderLine
                \multicolumn{2}{c|}{物理参数} & \multicolumn{4}{c}{力学参数} \\
                \borderLine
                \makecell[c]{密度/\\($\rm{g/cm^3}$)} & 吸水率 & \makecell[c]{杨氏模量($E$)/\\GPa} & \makecell[c]{拉伸强度($\sigma_t$)/\\MPa} & 断裂伸长率 & \makecell[c]{冲击强度/\\$\rm{kJ/m^2}$}    \\
                \interLine
                1.56 & 0.04$\sim$0.4 & 2.9$\sim$3.4 & 50$\sim$80 & 20$\sim$40\% & 2$\sim$5  \\
                \borderLine
            \end{tabular}
        }
        \end{center}
    \end{table}
    
    \begin{table}[!htb]
        \caption{通用 CPVC 的热学性能数据表}
        \label{tabCPVCTher}
        \begin{center}
        \footnotesize{
            \begin{tabular}{cccccc}
                \borderLine
                \multicolumn{6}{c}{热学参数}    \\
                \borderLine
                \makecell[c]{熔点($T_m$)/\\\cd} & \makecell[c]{玻璃化转变温度($T_g$)/\\\cd} & \makecell[c]{维卡软化点/\\\cd} & \makecell[c]{热导率/\\($\rm{W/(m\cdot K)}$)} & \makecell[c]{线膨胀系数($\alpha$)/\\K} & \makecell[c]{比热容($c$)/\\($\rm{kJ/(kg\cdot K)}$)} \\
                \interLine
                150 & 106$\sim$115 & 106$\sim$115 & 0.16 & $\rm{8 \times 10^{-5}}$ & 0.9    \\
                \borderLine
            \end{tabular}
        }
        \end{center}
    \end{table}
    
    \item 与其他塑料制品相比,CPVC 具有拉伸强度高、热膨胀系数小、热传导率低、氧指数高(难燃)、氧气透过率小等优良特性,具体数据见表 \ref{tabCompare}。
    
    \begin{table}[!htb]
        \caption{CPVC 管材与其他塑料管材主要力学性能对比\cite{9}}
        \label{tabCompare}
        \begin{center}
        \footnotesize{
            \begin{tabular}{cccccc}
                \borderLine
                塑料管材 & \makecell[c]{拉伸强度 \\ (23\cd)/MPa} & \makecell[c]{热膨胀系数 \\ $\rm{\times 10^4/K^{-1}}$} & \makecell[c]{热传导率/ \\ $[\rm{W/(m\cdot K)}]$} & \makecell[c]{氧指数/ \\ \%} & \makecell[c]{氧气透过量(70\cd、1个大气压)/ \\ $[\rm{cm^3/(m^2\cdot d)}]$}  \\
                \interLine
                CPVC 管材 & 55 & 0.7 & 0.14 & 60 & <1 \\
                PVC 管材 & 50 & 0.7 & 0.14 & 45 & <1  \\
                PP-R 管材 & 30 & 1.5 & 0.22 & 18 & 13$\sim$16 \\
                PE-X 管材 & 25 & 1.5 & 0.22 & 17 & 13 \\
                PB 管材 & 27 & 1.3 & 0.22 & 18 & 16   \\
                \borderLine
            \end{tabular}
        }
        \end{center}
    \end{table}
    
    \item 耐化学腐蚀性能好。工业化学药剂大都具有一定腐蚀性,而 CPVC 不仅在常温下具有优异的耐化学腐蚀性能,并且在较高温度下,仍能保持优于 PVC 及其他树脂的耐酸、耐碱、耐腐蚀性能。CPVC 可在许多方面取代传统的金属材料,用以应对需要直接接触腐蚀性物品的场合,如处理酸液、碱液、氧化性溶液以及浓盐溶液等。在恶劣的使用条件下,CPVC 仍能具有较长的使用寿命,维修成本低,并拥有良好的环境适应力。
    \item 阻燃自熄性能好。CPVC 的氧指数为 60,因此其在大气环境下不易燃烧,即便发生燃烧也不易产生滴落物,火焰扩散速度慢,烟雾少,毒性小。
    \item 很多聚烯烃材料(包括 PE、PP、PB 等)会与水中余氯发生反应而缓慢分解,而 CPVC 则不会受水中的余氯的影响,不会出现裂痕和崩漏\cite{17, 18}。
\end{enumerate}

\subsection{CPVC 降解机理}
在热加工过程中,PVC 极易发生降解脱除 HCl,其重要原因是 PVC 分子链中含烯丙基氯与叔碳氯等不稳定氯结构\cite{15}。其中烯丙基氯的含量更高,并且对 CPVC 脱 HCl 具有更强的催化效果。CPVC 与 PVC 具有相似的分子结构,分子链中也存在烯丙基氯和叔碳氯等不稳定氯结构,研究发现 CPVC 树脂的加工稳定性远不如 PVC\cite{6}。\par
\setatomsep{1.5em}
靖志国等人利用 $\rm{^{13}}$C NMR 对 CPVC 分子链序列结构的测定发现,CPVC 分子中氯原子沿碳链分布情况复杂,其分子链结构相当于氯乙烯、1,2-二氯乙烯以及 1,1-二氯乙烯的三元共聚物。CPVC 分子中主要结构的摩尔分数为:\chemfig{\va{C}-CHCl-} 含量为 65\%$\sim$70\%;\chemfig{\va{C}-CH_2-} 含量为 20\%$\sim$30\%;\chemfig{\va{C}-CCl_2-} 含量为 5\%$\sim$10\%。随着 $\omega_{Cl}$ 的增大,\chemfig{\va{C}-CHCl-} 和 \chemfig{\va{C}-CCl_2-} 两种含氯结构单元的含量增加,\chemfig{\va{C}-CH_2-} 结构单元减少。当 $\omega_{Cl}$ 提升到 65\% 以后,\chemfig{\va{C}-[@{op,.75}]CHCl-CHCl-[@{cl,.25}]} \makebraces[3pt,3pt]{}{op}{cl} 结构会使 CPVC 分子链的极性大大增加,从而 CPVC 分子链的刚性增强,使得 CPVC 的 $T_g$ 随之上升。\chemfig{\va{C}-CCl_2-} 结构的使得 CPVC 分子的极性下降,同时主链上的碳碳单键的旋转能垒下降,导致 CPVC 的 $T_g$ 下降;另外,\chemfig{\va{C}-CCl_2-} 结构的出现会降低 CPVC 分子的热稳定性,导致 CPVC 分子容易分解产生 HCl。因此,在 CPVC 的氯化过程中应尽量提高 \chemfig{\va{C}-[@{op,.75}]CHCl-CHCl-[@{cl,.25}]} \makebraces[3pt,3pt]{}{op}{cl} 的含量并降低 \chemfig{\va{C}-[@{op,.75}]CCl_2-CH_2-[@{cl,.25}]} \makebraces[3pt,3pt]{}{op}{cl} 的含量\cite{4}。\par

研究表明,CPVC 的热分解分为两步进行\cite{12}。第一步为脱除 HCl,生成 \chemfig{-[,1.5,,,decorate,decoration=snake]C([2]-Cl)=C([2]-H)-[,1.5,,,decorate,decoration=snake]} 和 \chemfig{-[,1.5,,,decorate,decoration=snake]C([2]=O)-[,1.5,,,decorate,decoration=snake]} 结构单元以及它们的共轭结构。第二步按 \textit{Diels–Alder} 机理发生缩合反应,随后生成具有多环芳香结构的含氯化合物。\par
目前,含氯聚合物脱除 HCl 的机理包括单分子机理、离子型机理和自由基机理。马文光等\cite{22}通过 ESR\footnote{电子自旋法}的研究结果表明,CPVC 最可能发生的是自由基机理脱 HCl。其过程为不稳定氯原子在热的作用下脱离形成 \lewis{0.,Cl}$\;$,\lewis{0.,Cl}$\;$ 进一步引发拉链式分解反应。如反应 \eqref{eqCPVCDegrade1} 至反应 \eqref{eqCPVCDegrade3} 所示。

\setpolymerdelim[]
\setatomsep{2em}

CPVC 分子中的结构缺陷,特别是烯丙基氯结构分解,产生 \lewis{0.,Cl}$\;$:
    \begin{equation}
    \small{
    \label{eqCPVCDegrade1}
    \schemestart
        \chemfig{\va{C}-[,1.2,,,decorate,decoration=snake]CH=CH-CH(-[2]Cl)-CH_2-[,1.2,,,decorate,decoration=snake]}
        \arrow(.mid east--.mid west)
        \chemfig{\va{C}-[,1.2,,,decorate,decoration=snake]CH=CH-\lewis{2.,C}H-CH_2-[,1.2,,,decorate,decoration=snake]}
        \makebraces[5pt,5pt]{}{op}{cl}
        \+
        \lewis{0.,Cl}
    \schemestop
    }
    \end{equation}

\lewis{0.,Cl}$\;$ 从 CPVC 分子中夺走 \lewis{0.,H}$\;$,形成链自由基。ESR 信号证明了大分子自由基的存在:
    \begin{equation}
    \small{
    \label{eqCPVCDegrade2}
    \schemestart
        \lewis{0.,Cl}
        \+
        \chemfig{\va{C}-[,1.2,,,decorate,decoration=snake]CH_2-CH([2]-Cl)-CH_2-CH([2]-Cl)-[,1.2,,,decorate,decoration=snake]}
        \arrow(.mid east--.mid west)
        \chemfig{\va{C}-[,1.2,,,decorate,decoration=snake]\lewis{2.,C}H-CH([2]-Cl)-CH_2-CH([2]-Cl)-[,1.2,,,decorate,decoration=snake]}
        \+
        \chemfig{HCl}
    \schemestop
    }
    \end{equation}

CPVC 链自由基脱除 \lewis{0.,Cl}$\;$,在分子链中形成双键。脱除的 \lewis{0.,Cl}$\;$ 促进反应 \eqref{eqCPVCDegrade2} 的发生,使两步反应回想促进,使 CPVC 发生链锁分解反应:
    \begin{equation}
    \small{
    \label{eqCPVCDegrade3}
    \schemestart
        \chemfig{\va{C}-[,1.2,,,decorate,decoration=snake]\lewis{6.,C}H-CH(-[2]{Cl})-CH_2-CH(-[2]{Cl})-[,1.2,,,decorate,decoration=snake]}
        \arrow(.mid east--.mid west)
        \chemfig{\va{C}-[,1.2,,,decorate,decoration=snake]CH=CH-CH_2-CH(-[2]{Cl})-[,1.2,,,decorate,decoration=snake]}
        \+
        \lewis{0.,Cl}
    \schemestop
    }
    \end{equation}

分子链末端的引发剂残基在热的作用下也会脱去形成自由基 \lewis{0.,R}$\;$,\lewis{0.,R}$\;$ 又引起进一步的链锁分解反应:
    \begin{equation}
    \small{
    \label{eqCPVCDegrade4}
    \schemestart
        \lewis{0.,R}
        \+
        \chemfig{\va{C}-[,1.2,,,decorate,decoration=snake]CH_2-CH(-[2]{Cl})-CH_2-CH(-[2]{Cl})-[,1.2,,,decorate,decoration=snake]}
        \arrow(.mid east--.mid west)
        \chemfig{\va{C}-[,1.2,,,decorate,decoration=snake]\lewis{2.,C}H-CH([2]-Cl)-CH_2-CH(-[2]{Cl})-[,1.2,,,decorate,decoration=snake]}
        \+
        \chemfig{RH}
    \schemestop
    }
    \end{equation}

在反应 \eqref{eqCPVCDegrade4} 之后,又会连续地发生反应 \eqref{eqCPVCDegrade3} 和反应 \eqref{eqCPVCDegrade2}。分解生成的大分子自由基也会发生链的转移,终止等反应形成支化、交联和不饱和双键结构。


\section{CPVC 的结构与合成工艺}

\subsection{CPVC 分子结构}
\setatomsep{1.5em}
CPVC 是 PVC 在热引发剂的作用下被 \chemfig{Cl_2} 氯化生成的产物。研究表明,在氯化反应中,氯原子优先进攻 PVC 分子链中的 \chemfig{\va{C}-CH_2-} 基团,而不是 \chemfig{\va{C}-CHCl-} 基团,因此得到的 CPVC 主要是 \chemfig{\va{C}-[@{op,.75}]CHCl-CHCl-[@{cl,.25}]} \makebraces[3pt,3pt]{}{op}{cl} 链节构成。当 CPVC 的 $\omega_{Cl}$ 低于 63\% 时,生成的结构大部分为 \chemfig{\va{C}-[@{op,.75}]CHCl-CHCl-[@{cl,.25}]} \makebraces[3pt,3pt]{}{op}{cl},该结构具有一定的偶极矩,使得分子间范德华力增强;当 $\omega_{Cl}$ 高于 63\% 时,才逐渐生成极性较小的 \chemfig{\va{C}-[@{op,.75}]CCl_2-CH_2-[@{cl,.25}]} \makebraces[3pt,3pt]{}{op}{cl} 结构。由此可见,PVC 的氯化反应主要发生在亚甲基碳原子上,生成 \chemfig{\va{C}-[@{op,.75}]CHCl-CHCl-[@{cl,.25}]} \makebraces[3pt,3pt]{}{op}{cl} 链节;其次再发生在次甲基碳原子上,生成 \chemfig{\va{C}-[@{op,.75}]CCl_2-CH_2-[@{cl,.25}]} \makebraces[3pt,3pt]{}{op}{cl} 链节。随着氯化程度的提高,\chemfig{\va{C}-[@{op,.75}]CCl_2-CH_2-[@{cl,.25}]} \makebraces[3pt,3pt]{}{op}{cl} 与 \chemfig{\va{C}-[@{op,.75}]CHCl-CHCl-[@{cl,.25}]} \makebraces[3pt,3pt]{}{op}{cl} 链节含量的比值增大。

\subsection{CPVC 树脂合成工艺}
目前,CPVC 树脂的合成工艺按氯化介质不同分为溶剂法、水相悬浮法和气固相法,其反应机理均为自由基反应。

\subsubsection{自由基氯化反应机理}
自由基氯化反应过程分为链引发、链传递、链终止三个阶段\cite{1},反应机理如反应 \eqref{eqChloride1}$\sim$\eqref{eqChloride5} 所示。\par

链引发反应:
    \begin{equation}
        \label{eqChloride1}
        \schemestart
            \chemfig{Cl_2}
            \arrow(.mid east--.mid west)
            2\lewis{0.,Cl}
        \schemestop
    \end{equation}

链传递反应:
    \begin{equation}
        \label{eqChloride2}
        \schemestart
            \lewis{0.,Cl}
            \+
            \chemfig{\va{C}-[,1.5,,,decorate,decoration=snake]C([2]-Cl)([6]-H)-C([2]-H)([6]-H)-[,1.5,,,decorate,decoration=snake]}
            \arrow(.mid east--.mid west)
            \chemfig{\va{C}-[,1.5,,,decorate,decoration=snake]C([2]-Cl)([6]-H)-C([2]-Cl)([6]-H)-[,1.5,,,decorate,decoration=snake]}
            \+
            \lewis{0.,H}
        \schemestop
    \end{equation}

    \begin{equation}
        \label{eqChloride3}
        \schemestart
            \lewis{0.,Cl}
            \+
            \chemfig{\va{C}-[,1.5,,,decorate,decoration=snake]C([2]-Cl)([6]-H)-C([2]-H)([6]-Cl)-[,1.5,,,decorate,decoration=snake]}
            \arrow(.mid east--.mid west)
            \chemfig{\va{C}-[,1.5,,,decorate,decoration=snake]C([2]-Cl)([6]-H)-C([2]-Cl)([6]-Cl)-[,1.5,,,decorate,decoration=snake]}
            \+
            \lewis{0.,H}
        \schemestop
    \end{equation}
    
    \begin{equation}
        \label{eqChloride4}
        \schemestart
            \lewis{0.,H}
            \+
            \chemfig{Cl_2}
            \arrow(.mid east--.mid west)
            \chemfig{HCl}
            \+
            \lewis{0.,Cl}
        \schemestop
    \end{equation}

链终止反应:
    \begin{equation}
        \label{eqChloride5}
        \schemestart
            2\lewis{0.,Cl}
            \arrow(.mid east--.mid west)
            \chemfig{Cl_2}
        \schemestop
    \end{equation}

常见的引发方式主要有单纯热引发、紫外光引发及低温等离子体引发。

\begin{enumerate}[(1) ]
    \item 单纯热引发方式即单纯依靠加热使 PVC 分子产生自由基从而制备 CPVC,所得产品的 $\omega_{Cl}$ 较低,反应过程中物料极易发黏变黄从而影响氯化反应的进行。
    \item 低温等离子体引发 PVC 氯化虽然能得到氯化均匀且具有较高 $\omega_{Cl}$ 的 CPVC,但是该引发方式制备 CPVC 较难实现工业化。
    \item 采用紫外光引发方式能够得到氯化均匀且具有较高 $\omega_{Cl}$ 的 CPVC,若能解决工程问题,有望实现工业化,以期解决目前 CPVC 生产工艺中存在的环境污染、产品后处理繁琐等弊端。
\end{enumerate}

\subsubsection{溶液法}

溶液法是 CPVC 生产中一种工艺相对成熟的方法。其主要流程如下:

\begin{enumerate}[(1) ]
	\item 用适当的溶剂对疏松型 PVC 进行预处理,使其与溶剂充分结合。
	\item 体系加热至 80$\sim$100\cd 时加入偶氮二乙腈作为引发剂。
	\item 向体系中通入 \chemfig{Cl_2} 并控制气体流量,反应得到 CPVC 树脂。
	\item 向溶剂中加入沉淀剂 \chemfig{CH_3OH} 使 CPVC 沉淀,并经过反复进行抽滤、洗涤,最终干燥得到 $\omega_{Cl}$ 为 64\%$\sim$75\% 的 CPVC 粉末产品。
\end{enumerate}

该法制得的 CPVC 树脂产品中氯原子的分布均匀,并且在有机溶剂中具有良好的溶解性,因此适合用来生产各种 CPVC 涂料及黏合剂。但由于得到的 CPVC 树脂为无规均质产品,其耐热性、力学性能较差,不能用于生产管材等硬质产品。同时,溶剂法由于需要使用大量 \chemfig{CCl_4}、\chemfig{C_6H_{13}Cl} 等有机溶剂,对环境污染较大且溶剂回收成本高,目前已逐渐减少使用。

\subsubsection{水相悬浮法}
水相悬浮法是以水或盐酸为分散介质,在搅拌作用下使 PVC 树脂悬浮于反应体系中。通入氮气充分除氧后,向反应体系中加入所需催化剂。缓慢向体系中通入氯气使反应开始进行,体系温度和压力上升。随着反应的进行,体系压力下降,至压力恒定时反应结束。泄压,在氮气保护下进行两次过滤、洗涤操作,滤饼进入稳定化工序\cite{23}。该法制得的非均质 CPVC 含氯量高、氯化均匀、热稳定性好。水相悬浮法制得的产品性能较好,但工艺流程较长,生产“三废”较多,成本相对较高。

\subsubsection{气固相法}
由张向京等\cite{1}报导的由气固相法制备 CPVC 的方法如下:

\begin{enumerate}[(1) ]
    \item 准确称取 5.0 g PVC 粉末置于流化床反应器中。
    \item 采用金属镀膜给反应器加热。料温达到 50$\sim$70\cd 时,通入 \chemfig{N_2} 防止 PVC 被氧化。持续升温至 80\cd 后,加大 \chemfig{N_2} 流量使物料流化,并保持料温稳定。
    \item 待达到氯化温度时,打开紫外灯\footnote{为充分活化 \chemfig{Cl_2} 并防止紫外光能量过高时造成 PVC 分解,选择能量相对适中的波长为 300 nm 的紫外光作为引发光源。} 并开始通入 \chemfig{Cl_2},通过调节 \chemfig{N_2} 与 \chemfig{Cl_2} 的流量来改变原料气中的 $\varphi_{Cl_2}$\footnote{$\varphi_{Cl_2}$: \chemfig{Cl_2} 的体积分数},尾气用 KOH 吸收。
    \item 反应结束后,用蒸馏水浸泡样品 0.5 h,抽滤,重复操作至中性后于 60\cd 真空干燥至恒重。
\end{enumerate}

用该法生产 CPVC,具有流程简单、污染物排放小的优点,但由于传热效果较差,不适宜大规模生产。


\section{CPVC 树脂加工与改性}
不同用途和性能的 CPVC 制品其配方设计不同,但是其基本配方都含有热稳定剂、润滑剂及其他助剂(如加工改性剂、冲击改性剂、填料、光稳定剂、着色剂、抗静电剂等)。

\subsection{热稳定剂}
CPVC 的热稳定性差,同时因其氯含量高、分子刚性大,CPVC 的加工性能较 PVC 更差,更易分解释放出 HCl,导致制品容易变色、变硬和烧焦。热稳定剂是一类能防止或减缓聚合物在加工使用过程中因受热而发生降解或交联,从而延长材料使用寿命的添加剂\cite{27}。\par
热稳定剂可与烯丙基氯原子结合、吸收分解产生的 HCl、与共轭双键发生加成反应、捕捉自由基等方式抑制 CPVC 分子的分解。热稳定剂的稳定机理主要有以下 4 种:

\begin{enumerate}[(1) ]
    \item 吸收或中和 CPVC 分解释放的 HCl,抑制 HCl 对 CPVC 分解的催化作用。大部分热稳定剂都具有该作用。
    \item 结合并置换 CPVC 中的活泼烯丙基氯使其失活从而阻止 CPVC 分解产生 \lewis{0., Cl}\;。如有机锡热稳定剂会与 CPVC 分子中的烯丙基氯结构发生配位结合,使其不易分解。
    \item 与共轭多烯结构发生 \textit{Diels–Alder} 反应,从而破坏 CPVC 中的大 $\pi$ 键,减少变色。该类主要包括不饱和脂肪酸的盐和酯。
    \item 捕捉 CPVC 分解产生的 \lewis{0., Cl}\; 和大分子链自由基,从而阻止氧化反应。如酚类热稳定剂能够分解产生 \lewis{0., H}\;,与 \lewis{0., Cl}\; 和大分子链自由基发生偶合,从而提高 CPVC 的热稳定性
\end{enumerate}

目前人们做的比较多的是对铅盐类热稳定剂和有机锡热稳定剂的研究。\par
马玫等\cite{6}研究了复合铅体系对加工稳定性的影响。研究结果表明:将三盐基硫酸铅与二盐基亚磷酸铅复合使用能提供比单独使用更好的稳定性,与使用有机铅盐相近。铅盐稳定剂复合使用,对 CPVC 确有协同效应,6 份的铅盐稳定剂可以满足 CPVC 的加工需求。对于共稳定剂,亚磷酸盐能与铅盐稳定剂产生协同效应,随着亚磷酸酯的用量增大,CPVC 的塑化温度和平衡扭矩显著减小,提高了塑化效果的同时也使其性能(维卡软化温度、冲击强度)得到了提高。\par
柯伟席等\cite{8}研究了二盐、三盐、有机锡稳定剂和复合铅盐类热稳定剂以及辅助稳定剂对 CPVC 热稳定性及加工性能的影响。研究结果表明:复合铅盐类热稳定剂的稳定效果最好,且使得 CPVC 更易于加工。对于辅助热稳定剂,发现 Pb-St 和 Ba-St 具有良好的长期热稳定性及润滑性,并且两者具有协同作用。随着辅助热稳定剂用量的增加,CPVC 的塑化时间呈先上升后下降的趋势,Pb-St 和 Ba-St 的用量为 0.7$\sim$1.1 份比较合适。

\subsection{润滑剂}
CPVC 树脂比 PVC 树脂具有更高的熔体黏度,因此其加工性能更差。润滑剂是一类用于降低熔体黏度以及熔体与设备之间的摩擦力,从而延长树脂的热稳定时间以及减少设备的磨损与能耗的加工助剂。\par
毛季红\cite{2}研究了内外润滑剂的品种和用量对 CPVC 材料流变性能的影响。该研究表明:在相同用量下,微晶石蜡和酰胺蜡作为外润滑剂具有最长的塑化时间以及最低的平衡转矩,因此认为其能够产生相对较好的润滑效果。进一步使用微晶石蜡和酰胺蜡,控制组成不变的情况下改变用量,发现外润滑剂用量越多则 CPVC 树脂的热稳定时间越长,且平衡转矩也越低,说明须足够的润滑剂用量\footnote{从实验结果来看,外润滑剂用量须在 2.5$\sim$3.0 份之间}才能保证较好的挤出效果。对于内润滑剂的研究发现,多元醇酯和脂肪酸酯的润滑效果相近,同时增加内润滑剂的用量可以促进树脂塑化。但内润滑剂熔点大都较低,其在加热过程中较早熔化,会降低 CPVC 树脂的热稳定性,因而需对其用量进行控制。

\subsection{抗冲改性剂}
随着 CPVC 分子中 $\omega_{Cl}$ 的增大,分子链的刚性增强,并且大分子链间的范德华作用也由于 \chemfig{C-Cl} 键的增多而增大,最终导致 CPVC 的脆性相比于 PVC 更大,制品的抗冲击性能变差。考虑到 CPVC 制品在加工、运输以及使用过程中可能受到的冲击作用,需在 CPVC 加工过程中加入适量的抗冲改性剂,从而提高 CPVC 的塑化效果以及改善其在低温下的韧性。\par
目前在 CPVC 制品的生产中,普遍使用 CPE\footnote{氯化聚乙烯}、MBS\footnote{(甲基丙烯酸甲醋-丁二烯-苯乙烯)共聚树脂}、ACR\footnote{抗 冲型丙烯酸醋橡胶}、ABS\footnote{丙烯睛-丁二烯-苯乙烯} 对 CPVC 进行改性。作为常用的抗冲改性剂,MBS、ACR 都具有典型的“核-壳”结构,都可在低用量下明显改善 CPVC 制品的脆性以及加工性能\cite{24}。

\subsection{其他助剂}
除以上助剂外,CPVC 在加工过程中还应根据具体的加工需求加入填料、光稳定剂、着色剂、抗静电剂等助剂。合适比例的填料可在降低生产成本的同时增强塑料制品的强度。


\section{CPVC 树脂的应用}
与其他热塑性塑料相比,CPVC 具有更优异的耐热性能、耐腐蚀性能和阻燃性能,并且具有更低的成本。基于以上优势,CPVC 被广泛应用于生产加工各种管件管材、建筑装潢材料、电子电器设备配件等\cite{5}。下面介绍几个 CPVC 的应用领域。

\subsection{管件、管材}
由于 CPVC 对严重腐蚀(强酸、强碱、有机溶剂、浓盐溶液等)和对恶劣环境侵蚀的优良耐受力,被广泛应用于冶炼工业、石油化工、造纸业、电镀业以及民用排污管道。在民用方面,CPVC 管材被广泛应用于民用及服务大楼的供热水管。使用 CPVC 冷热水管可提供一套清洁、安全、耐热、耐腐蚀、阻燃、易于安装的管道系统。

\subsection{阻燃材料}
CPVC 具有优异的阻燃性能及消烟性能,使其可替代其他塑料材料成为阻燃塑料产品的首选,可应用于需耐高温的建筑房屋的阻燃隔层、电子电器设备配件、消防设备等领域,具有十分广阔的应用前景。

\subsection{涂料和黏合剂}
将 CPVC 溶于丙酮、氯代烃等有机溶剂中,可制得具有广泛用途的涂料和黏合剂。CPVC 涂料可应用于机械设备的防腐层、纤维或其复合材料的阻燃层、建筑外墙涂层等领域。CPVC 树脂具有优良的耐化学药品性、阻燃性以及耐热性,同时具有低的氧气透过量\footnote{表 \ref{tabCompare}},使其成为最重要的防腐涂料品种之一。CPVC 黏合剂具有粘黏结强度高、耐腐蚀性能好等优点,可用来黏合各种 PVC 和 CPVC 的管件。

\subsection{电力电缆用 CPVC 护套}
电力电缆长期埋在地下或在室外环境使用,承受恶劣的环境考验,因此要求其具有优良的耐候性、耐热性、绝缘性以及阻燃性。同时 CPVC 护套具有较其他树脂更高的维卡软化温度,可承受电力传输过程中因焦耳效应产生的高温。

\section{热稳定剂概述}

在 CPVC 的加工中常用的热稳定剂种类包括铅盐类热、金属皂类、有机锡类和稀土类。

\subsection{铅盐类热稳定剂}
铅盐类是一类最早投入使用的热稳定剂,在较低的生产成本下仍能提供很好的稳定效果。其中最常使用的包括 \chemfig{3PbO·PbSO_4·H_2O}、\chemfig{2PbO·PbHPO_3}、\chemfig{2PbO·Pb{(C_{17}H_{35}COO)}_2} 和 \chemfig{2PbCO_3·Pb{(OH)}_2} 等碱式铅盐。铅盐稳定剂具有良好的热稳定效果,并且不会影响制品的介电性能。当与适当的润滑剂共同使用时,可提高 CPVC 的加工温度上限,并且最终的制品具有良好的质量,因此广泛应用于 CPVC 的加工中。但由于 Pb 的生物毒性,铅盐稳定剂不能应用于与食品接触的制品加工。稳定机理:Pb 具有吸收 HCl 的能力,且生成的 \chemfig{PbCl_2} 不会催化 CPVC 的分解。

\subsection{金属皂类热稳定剂}
金属皂是指高级脂肪酸的金属盐,CPVC 的金属皂类热稳定剂一般为 Pb、Ba、Ca、Cd、Zn、Mg 等金属元素与硬脂酸、油酸等高级脂肪酸组成\cite{5}。根据其对 CPVC 稳定作用机理的不同,可将金属皂热稳定剂分为两大类:一类是主金属皂稳定剂,以 Cd 和 Zn 为金属基,能够吸收 CPVC 分解释放的 HCl,并且与 CPVC 分子链中的烯丙基氯原子发生置换从而抑制 \lewis{0., Cl}\; 的产生。但其生成的金属氯化物为 \textit{Lewis} 酸,会促进脱 HCl 的反应。另一类为辅金属皂催化剂,金属皂基为 Ba、Sr、Ca、Mg 等 \uppercase\expandafter{\romannumeral2}A 族金属元素。其仅仅显示出了捕捉 HCl 的作用,但其生成的金属氯化物无催化作用。因此在实际生产加工中,往往会将两种金属皂配合使用,从而发挥其协同效果。

\subsection{有机锡热稳定剂}

有机锡是一种热稳定性较好的热稳定剂,其透明性好且大多无毒。有机锡热稳定剂可分为烷基锡和酯基锡两大类。其中烷基锡包括四烷基锡和二烷基锡二氯化物等,酯基锡中的主要产品为硫醇酯基锡。有机锡稳定剂对 CPVC 分解释放出的 HCl 具有较好的吸收作用,同时硫醇甲基锡可通过特殊的配位作用,取代 CPVC 分子上的烯丙基氯原子\cite{29}。研究发现,有机锡稳定剂与辅助稳定剂或金属皂类稳定剂配合使用时,可提高有机锡的稳定效果,同时能阻止迁出、降低成本\cite{30}。

\subsection{辅助型热稳定剂}
为了提高金属皂类热稳定剂的效果,降低有机锡类的成本,人们研发出了多种多样的辅助型热稳定剂。这类稳定剂不含金属元素,因此也被称为非金属稳定剂。辅助热稳定剂与金属稳定剂配合使用时,能发挥较好的协同效果。目前广泛使用的辅助热稳定剂主要包括亚磷酸酯、环氧化合物、多元醇、含氮化合物、含硫化合物、$\beta$-二酮化合物等\cite{26}。

\section{润滑剂概述}

润滑剂从实现的功能上进行分类,可分为外润滑剂和内润滑剂。

\subsection{外润滑剂}
外润滑剂是由极性较小的长碳链分子组成,因此其在加工过程中,易从 CPVC 分子间迁出,并在 CPVC 与金属加工设备内壁间形成润滑膜,从而改善熔体的剥离性。并且能有效的改善 CPVC 的流变学性能、降低加工能耗。但外润滑剂会使材料的力学性能降低,因此在实际加工过程中,外润滑剂的用量要尽量小。常用的外润滑剂有固体石蜡、聚乙烯蜡、氧化聚乙烯蜡等极性较小的聚烯烃及其衍生物。

\subsection{内润滑剂}
内润滑剂与聚合物的相容性较好,因此其可以渗入到 CPVC 分子之间,起到降低分子作用力的效果。同时,内润滑剂与 CPVC 分子链之前结合作用不强,因此可以产生类似滚动和滑移的作用,减小分子间的摩擦力。内滑润剂可有效降低 CPVC 熔体的黏度,缓解熔体破裂和出膜膨胀现象。加入内润滑剂对塑化时间以及产品透明性的影响不大,但会降低材料的维卡软化点。常用的内润滑剂主要是脂肪酸酰胺类极性大的分子。
\chapter{实验部分}

\section{实验原料}

\begin{table}[!htb]
    \caption{实验原料明细表}
    \label{tabRaw}
    \begin{center}
        \begin{tabular}{ccc}
             \borderLine
             原料名称 & 型号 & 生产厂商		\\
             \interLine
             CPVC 树脂 & 工业级 & 上海氯碱化工股份有限公司	\\
			 PVC 树脂 & SG-5 & 上海氯碱化工股份有限公司		\\
% 			 \cline{1-1}
			 \multirow{3}{*}{有机锡稳定剂} & T-190A & 法国阿科玛公司	\\
			 & TMG-234 & \\
			 & 液体有机锡 & 山东祥生塑胶有限公司\\
% 			 \cline{1-1}
			 \multirow{4}{*}{外润滑剂} & AC-316 & \textit{Honeywell}	\\
			 & AC-617 & \textit{Honeywell}	\\
			 & AC-629 & \textit{Honeywell}	\\
			 & PEW-0380 &	\\
% 			 \cline{1-1}
			 内润滑剂 & 汉高 G-60 &	\\
			 抗冲改性剂 & MB838A &	\\
			 加工助剂 & 罗门哈斯 K-175P &	\\
			 颜料 & 钛白粉 &	\\
             \borderLine
        \end{tabular}
    \end{center}
\end{table}

外润滑剂使用 \textit{Honeywell} 公司的 AC-316、AC-617、AC-629,具体参数见表 \ref{tabSmootherHoney}。

\begin{table}[!htb]
	\caption{AC-316、AC-617、AC-629 的物理参数对比}
	\label{tabSmootherHoney}
	\begin{center}
		\begin{tabular}{cccc}
				\borderLine
				产品型号 & 密度/(g/$\rm{cm^3}$) & 滴点/\cd & 黏度 @ 140\cd/Pa$\cdot$s   \\
				\interLine
				AC-316 & 0.98 & 140 & 8500 \\
				AC-617 & 0.91 & 101 & 180  \\
				AC-629 & 0.93 & 101 & 200  \\
				\borderLine
		\end{tabular}
	\end{center}
\end{table}


\section{实验主要设备及仪器}

\begin{table}[H]
	\caption{实验仪器明细表}
	\label{tabEqu}
	\begin{center}
		\begin{tabular}{ccc}
				\borderLine
				仪器名称 & 型号 & 生产厂商		\\
				\interLine
				高速混合机 & SHR & 张家港市永利机械有限公司	\\
				开放式炼胶机 & XK-160 & 无锡明达橡塑机械有限公司	\\
				平板硫化机 & QLB-D & 上海橡胶机械厂	\\
				电子万能材料试验机 & INSTRON5567 & 美斯特工业系统(中国)有限公司	\\
				电子天平 & AR153CN & 奥豪斯仪器(上海)有限公司	\\
				真空干燥箱 & DZF-6050型 & 上海一恒科技有限公司	\\
				HAAKE 转矩流变仪 & Polylab OS & 德国HAAKE公司	\\
				热重分析仪 & TGA Q50 & 美国TA公司	\\
				数字测厚仪 & GH-01 & 广州标际包装设备有限公司	\\
				扫描电子显微镜 & S-4700 & 日本日立公司	\\
				塑料摆锤冲击试验机 & ZBC & 美特斯工业系统(中国)有限公司	\\
				维卡软化点试验机 & ZWK & 美特斯工业系统(中国)有限公司	\\
				动态力学分析仪 & DMA Q800 & 美国TA公司	\\
				\borderLine
		\end{tabular}
	\end{center}
\end{table}

\section{实验流程}

\begin{figure}[!htb]
    \begin{center}
		\begin{tikzpicture}[node distance = 2cm]
			% nodes
			\node(smooth)[io]{润滑剂};
			\node(stablizer)[io, below of = smooth]{热稳定剂};
			\node(p1)[coordinate, right of = smooth, yshift = -1cm]{};
			\node(pre)[pro, right of = p1]{配方设计};
			\node(sample)[pro, below of = pre]{制样};
			\node(exp)[pro, below of = sample]{性能测试};
			\node(p2)[coordinate, right of = exp, xshift = 1cm]{};
			\node(bend)[pro, right of = p2]{弯曲强度};
			\node(tensile)[pro, above of = bend, node distance = 1.5cm]{拉伸强度};
			\node(dynamic)[pro, above of = tensile, node distance = 1.5cm]{动态热稳定性};
			\node(loss)[pro, above of = dynamic, node distance = 1.5cm]{热失重分析};
			\node(static)[pro, above of = loss, node distance = 1.5cm]{静态热稳定性};
			\node(impact)[pro, below of = bend, node distance = 1.5cm]{冲击强度};
			\node(SEM)[pro, below of = impact, node distance = 1.5cm]{扫描电镜测试};
			\node(DMA)[pro, below of = SEM, node distance = 1.5cm]{动态力学热分析};
			\node(vic)[pro, below of = DMA, node distance = 1.5cm]{维卡软化点};
			\node(result)[io, below of = exp]{实验数据分析与结论};
			
			% arrows
			\draw[arrow](smooth.east) -| (p1) -- (pre);
			\draw[arrow](stablizer.east) -| (p1) -- (pre);
			\draw[arrow](pre.south) -- (sample);
			\draw[arrow](sample.south) -- (exp);
			\draw[line](exp.east) -- (p2);
			\draw[arrow](p2) |- (tensile);
			\draw[arrow](p2) |- (dynamic);
			\draw[arrow](p2) |- (loss);
			\draw[arrow](p2) |- (static);
			\draw[arrow](p2) |- (impact);
			\draw[arrow](p2) |- (SEM);
			\draw[arrow](p2) |- (DMA);
			\draw[arrow](p2) |- (vic);
			\draw[arrow](p2) -- (bend);
			\draw[arrow](exp.south) -- (result);
		\end{tikzpicture}
    \end{center}
    \caption{研究方案流程图}
\end{figure}

\subsection{配方设计}
CPVC 润滑体系与热稳定体系采用控制变量法进行配方设计,配方的基本组成如表 \ref{tabCPVCFormula} 所示

\begin{table}[!htb]
    \caption{CPVC 基本配方设计表}
    \label{tabCPVCFormula}
    \begin{center}
    \footnotesize{
        \begin{tabular}{ccccccc}
            \borderLine
            CPVC & 抗冲击改性剂 & 热稳定剂 & 外润滑剂 & 内润滑剂 & 加工助剂 & 钛白粉 \\
            \interLine
            100\footnotemark[1] & 8 & 2 & 1.3 & 1.2 & 3 & 2   \\
            \borderLine
        \end{tabular}
    }
    \end{center}
\end{table}
\footnotetext[1]{均表示份数}

\subsection{制样}

\subsubsection{混料}
配混料混合效果的好坏将直接影响制品的均匀性与力学性能,因而该步采用高速混合机进行原料的混合。\par
按配方准确称量各助剂,将 CPVC 树脂及各助剂按顺序加入到高速混合机中,混合 3 分钟制成配混料。为防止因摩擦生热使得 CPVC 发生热分解,采用每搅拌 2 s 暂停 3 s 的间歇式搅拌方法,控制在树脂在较低的温度。

\subsubsection{塑化开炼}
该步是将制得的配混料加入到双辊开炼机中,通过控制两个辊筒的转速比使得物料受到剪切作用,从而达到塑炼的混合效果。\par
通过对 CPVC 玻璃化转变温度 $T_g$ 与热分解温度 $T_d$ 的参考,将双辊开炼机的辊温设定为 190\cd。将配混料加入到开炼机中反复进行塑化开炼,塑化时间约 5 min。

\subsubsection{压片}
在 180\cd、10 MPa 条件下,采用平板硫化机热压 3 min 左右,重复开合压板排气 4$\sim$5 次,再冷压 5 min 即得到待测试样片。

\subsubsection{切割}
按照最终性能测试的要求,将样片切割成标准样条,并对标准实验进行热学和力学性能的测试。


\section{性能测试}
CPVC 作为一种含氯聚合物,极易在受热时发生热分解。在加工过程中,由于 CPVC 长期处于受热状态,或在较高的使用温度下都会使得 CPVC 材料发生老化与热分解。同时,受热状态又可分为静态受热与动态受热,因此采用静态热稳定性测试和动态热稳定性测试分别对 CPVC 的热稳定性进行表征。同时用玻璃化转变温度和维卡热变形温度对 CPVC 在高温下的性能进行表征。\par
对于力学性能采用国家标准对 CPVC 的拉伸强度、弯曲强度、冲击强度进行测试,同时采用 SEM\footnote{扫描电子显微镜} 对冲击断面的形貌进行观测。

\subsection{热老化试验箱法测试}
CPVC 配混料在加工或再加工过程都会在较高温度的设备中停留一定时间,CPVC 制品在使用过程中也会经受一定的环境温度,这就要求热稳定剂能赋予 CPVC 以合适的静态热稳定性。根据 CPVC 热分解导致物料颜色变化或释放出氯化氢的特征,建立了变色法和脱氯化氢法两类评价静态热稳定性的方法。本实验采用变色法\footnote{执行标准 GB/T 9349—2002《聚氯乙烯、相关含氯均聚物和共聚物及其共混物热稳定性的测定 变色法》}进行 CPVC 静态热稳定性的表征。\par
烘箱法:将边长 15 mm,厚度约 1 mm 的正方形试样,放在平铺于架子上的新的干净铝箔上面,在强制鼓风烘箱中于高温下加热不同时间。每隔一定时间取出一片试样,从测试开始至最初观察到颜色变化即为初期热稳定定性,从测试开始至试样完全变黑的时间则为长期热稳定性。

\subsection{热失重分析测试}

\subsection{动态热稳定性测试}\label{sectionHakee}
动态热稳定性是指在热、空气和剪切力的共同作用下,热稳定剂抵抗 CPVC 热分解的能力。本实验采用转矩流变仪进行测试。首先将流变仪的温度设定为 205\cd,转速为 50 r/min。将 60 g 的样品加入到流变仪中,记录流变仪的转矩随时间的变化。在最终得到的 转矩-时间 曲线中,如图 \ref{figExHakee} 所示,笫一个峰为熔化峰,所对应的转矩为熔化转矩(fusion torgue)。熔化峰之后,由于物料进一步塑化并且熔体温度上升,使试样转矩下降,并随着熔体温度趋于恒定,转矩曲线呈现基本平稳段,所对应的转矩称为熔体转矩(melt torgue),也称平衡转矩。平衡转矩可用于评定样品的加工性能,平衡转矩越小,样品加工时所需的能耗越小,对设备的损耗也越小。随着试验的继续,CPVC 发生分解,此时曲线急速上升,混合物的长期热稳定性就是根据从熔化峰到转矩突然增大点所经历的时间来评定。

\begin{figure}[!htb]
    \begin{center}
        \input{src/example/hakee.tex}
    \end{center}
    \caption{CPVC 转矩-时间 曲线示例图}
    \label{figExHakee}
\end{figure}

\subsection{拉伸性能测试}
拉伸强度定义为断裂前试样所能承受的最大应力,单位为 MPa,用来评价材料的抗拉性能。拉伸强度的计算公式见式 \eqref{eqTenS},其中 $P$ 为样品承受的最大载荷,$b$ 和 $d$ 分别为试样的宽度和厚度。

\begin{equation}
    \label{eqTenS}
    \sigma_t = \frac{P}{bd}
\end{equation}

本实验中采用万能试验机进行拉伸强度测试\footnote{执行标准 GB/T 1040.2-2006},设置拉伸速率为 10 mm/min,夹具距离为 80 mm,样条的最窄宽度为 6 mm,厚度为 4 mm。

\subsection{弯曲性能测试}
弯曲强度是指材料在弯曲负荷作用下破裂或达到规定弯矩时能承受的最大应力,此应力为弯曲时的最大正应力,以 MPa 为单位。它反映了材料抗弯曲的能力,用来评价材料的弯曲性能。横力弯曲时,弯矩 M 随截面位置变化,一般情况下,最大正应力 $\sigma_{max}$ 发生于弯矩最大的截面上,且离中性轴最远处。因此,最大正应力不仅与弯矩 M 有关,还与截面形状和尺寸有关。最大正应力计算公式见式 \eqref{eqBendS},其中 $\sigma_{max}$ 为最大弯矩,$W$ 为抗弯截面系数。

\begin{equation}
    \label{eqBendS}
    \sigma_{max} = \frac{M_{max}}{W}
\end{equation}

本实验同样使用万能试验机进行弯曲强度测试\footnote{执行标准 GB/T 9341-2008},设置移动速率为 2 mm/min,样条尺寸为 80 mm$\times$10 mm$\times$4 mm,跨度为 64 mm。

\subsection{冲击性能测试}
冲击强度是材料在受到冲击后断裂吸收冲击能量的能力,用于评价材料的抗冲击能力或判断材料的脆性和韧性程度。缺口冲击强度的计算公式见式 \eqref{eqImpactS},其中 $aiN$ 为缺口冲击强度(Izod impact strength of a notched specimen),$x\%$ 为实验测得百分比,$S$ 为缺口处截面面积。

\begin{equation}
    \label{eqImpactS}
    aiN = (\frac{2.57 J \times x\%}{S}) KJ/m^2
\end{equation}

本实验采用落锤冲击强度仪进行缺口冲击强度测试\footnote{执行标准 ISO 180/1A},缺口形状为“V”形,深度为 2 mm,落锤满载能量为 2.75 J。

\subsection{扫描电镜测试}
SEM 的最大特点是图像富有立体感,放大倍数连续可变,特别适合表面形态的研究,是研究固体材料表面三维结构形态的有效工具,成为常用的高分子表面形貌剖析手段。

\subsection{动态力学热分析(DMA)}
本实验中采用 DMA 对玻璃化转变温度进行测试。DMA 是对试样施加恒定振幅的正弦交变应力,使其发生受迫振动,观察应变随温度或时间的变化规律,从而计算力学参数用以表征材料粘弹性的一种试验方法。在聚合物玻璃化转变过程中,其粘弹性有很大变化,从而可用 DMA 测定 $T_g$。DMA 曲线通常有储能模量、损耗模量、损耗因子这三个信号,对应的 $T_g$ 也可有三种取法,分别为储能模量的台阶式下降曲线部分的起始点、损耗模量的峰值温度、损耗因子\footnote{损耗角正切:$tan \, \delta = \frac{G''}{G'}$}的峰值温度。本实验取损耗因子的峰值温度作为最终测试得到的玻璃化转变温度。如图 \ref{figExTg} 所示,在加热过程中,样品的损耗因子出现了一个峰值,取峰值所在温度为样品的玻璃化转变温度。

\begin{figure}[!htb]
    \begin{center}
        \input{src/example/tg.tex}
    \end{center}
    \caption{CPVC DMA 损耗因子-温度 曲线示例图}
    \label{figExTg}
\end{figure}

\subsection{维卡软化点测试}
维卡软化点是将热塑性塑料置于特定液体传热介质中,在一定的负荷、一定的等速升温条件下,测定试样被 1 $\rm{mm^2}$ 针头压入 1 mm 时的温度\footnote{执行标准 GB 1633-1979}。实验测得的维卡软化点适用于控制质量和作为衡量材料热性能的一个指标,但不代表材料的使用温度。
\include{chapters/results}
\chapter*{致谢}
感谢


\clearpage
\addcontentsline{toc}{chapter}{参考文献}
\bibliographystyle{unsrtnat}    %unsrtnat: sort with cite order; plainnat: sort with author name and year
\bibliography{biblio.bib}

\end{document}
